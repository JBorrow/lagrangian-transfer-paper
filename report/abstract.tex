
%A framework for analysing the movement of baryonic matter in cosmological
%simulations is presented, requiring only the initial conditions and final
%state of the simulation. This is performed using two independent metrics; the
%distance between two closest neighbours in the initial conditions in the
%final simulation state, and the fractions of mass in halos at redshift $z=0$
%that originated in a given lagrangian region as defined by the dark matter.
%The framework is then applied to the $50\hmpc{}$, $512^3$ particle \simba{}
%box, showing that only 60\% of the baryonic matter in a given halo at $z=0$,
%roughly independent of halo mass, originates from the lagrangian region
%defined by the resident dark matter in that halo. This presents possible
%issues for semi- analytic models of galaxy formation, as well as zoom-in
%simulations, as up to 20\% of the baryonic mass in a typical Milky Way mass
%halo originated in the region defined by the dark matter of another halo.


A framework for characterizing the large scale movement of baryons relative
to dark matter in cosmological simulations is presented, requiring only the
initial conditions and final state of the simulation. This is performed using
the {\it spread metric}, quantifying the distance in the final conditions
between initial neighbouring mass resolution elements, and analysing the
baryonic content of final halos relative to that of the initial Lagrangian
regions defined by their dark matter component. Applying this framework to
the \simba{} cosmological simulations, we show that 40\% (10\%) of
cosmological baryons have moved $> 1$\,Mpc (3\,Mpc) by $z=0$, owing primarily
to jets powered by AGN, with baryons moving up to 12\,Mpc away in extreme
cases. Baryons decouple from the dynamics of the dark matter component owing
to hydrodynamic forces, radiative cooling, and feedback processes. As a
result, only 60\% of the gas content in a given halo at $z=0$ originates from
its Lagrangian region, roughly independent of halo mass, while a typical halo
in the mass range $M_{\rm vir} = 10^{12}$--$10^{13}\msolar$ only retains 20\%
of the gas originally contained in its Lagrangian region. We show that up to
20\% of the gas content in a typical Milky Way mass halo may originate in the
region defined by the dark matter of another halo. This {\it inter-Lagrangian
baryon transfer} may have important implications for semi-analytic models of
galaxy formation, ``zoom-in" simulations, and...