We present a framework for characterizing the large scale movement of baryons
relative to dark matter in cosmological simulations, requiring only the
initial conditions and final state of the simulation. This is performed using
the {\it spread metric} which quantifies the distance in the final conditions
between initially neighbouring particles, and by analysing the baryonic
content of final haloes relative to that of the initial Lagrangian regions
defined by their dark matter component. Applying this framework to the
\simba{} cosmological simulations, we show that 40\% (10\%) of cosmological
baryons have moved $> 1\hmpc{}$ ($3\hmpc{}$) by $z=0$, due primarily to
entrainment of gas by jets powered by AGN, with baryons moving up to
$12\hmpc{}$ away in extreme cases. Baryons
decouple from the dynamics of the dark matter component due to hydrodynamic
forces, radiative cooling, and feedback processes. As a result, only 60\% of
the gas content in a given halo at $z=0$ originates from its Lagrangian
region, roughly independent of halo mass. A typical halo in the mass range
$M_{\rm vir} = 10^{12}$--$10^{13}\msolar$ only retains 20\% of the gas
originally contained in its Lagrangian region. We show that up to 20\% of the
gas content in a typical Milky Way mass halo may originate in the region
defined by the dark matter of another halo. This {\it inter-Lagrangian baryon
transfer} may have important implications for the origin of gas and metals in
the circumgalactic medium of galaxies, as well as for semi-analytic models of
galaxy formation and “zoom-in" simulations.