In this work a new framework for analysing the movement of baryonic matter in
cosmological simulations is presented. This analysis uses only the initial
conditions and final state of the simulation to look at how matter has been
transferred between bound structures. This is performed using two independent
metrics; the distance between two closest neighbours in the initial conditions
in the final simulation state, and the fractions of mass in halos at redshift
$z=0$ that originated in a given lagrangian region as defined by the dark
matter. Only 60\% of the baryonic matter in a given halo at $z=0$, roughly
independent of halo mass, originates from the lagrangian region defined by the
resident dark matter in that halo. The fraction of baryonic mass in a given
halo, as a  function of radius, from inside the lagrangian region, from
outside, and from other lagrangian regions is a very well constrained function
providing insights onto the assembly of the Circumgalactic Medium (CGM). This
presents interesting problems for semi-analytic models of galaxy formation, as
well as zoom-in simulations, as at least 10\% of the baryonic mass in a halo
originated in the region defined by the dark matter of another halo. We present
a number of significant avenues for future research.