\section{Ensuring the method is robust}
\label{sec:robustness}

The above results are quite striking; that only around 50\% of the gaseous
mass in a given halo originates from the halo's own lagrangian region calls
for a healthy dose of skepticisim. There are a number of parameters, discussed
below, where different methodologies and definitions can be applied to see
if the results are robust.

\subsection{Variations in virial radius, \rvir{}}

There is no particular, well defined, reason that the extension of the particles
in a halo to the corresponding lagrangian region should end at the virial radius
of that halo. In a similar fashion, there is not just a single definition of the
`virial radius'; in this section chagning the radius at which particles are
selected to be part of the eventual lagrangian region is explored.

The procedure for extending the lagrangian region is as follows:
\begin{itemize}
    \item For every halo in the box, search for the centre of that halo (by
          looking for the extreme particles in each direction and finding
          their centre point, as well as taking into account the periodic
          boundaries), and the corresponding radius.
    \item Multiply this radius by a factor, such as 1.2, or 1.5
    \item For each halo in the box, use a periodic KDTree to search for the
          neighbours of the centre point within that radius, going from
          highest mass (in dark matter) to lowest mass to ensure that
          lower-mass halos `steal' from the higher mass ones, should they
          be embedded or nearby.
    \item Label these particles as belonging to the halo
    \item Re-run the original analysis with these halo definitions
\end{itemize}

In Figure \ref{fig:comparevirialradii}, the mass fraction mass functions are
shown.

\begin{figure}
    \centering
    \includegraphics{}
    \caption{}
    \label{fig:comparevirialradii}
\end{figure}

I don't actually know what these results will be, we'll have to see!

\subsection{The definition of lagrangian region}

In the above analysis, we considered a very diffuse notion of a lagrangian
region, defined particle-by-particle. Whilst increasing the virial radius
will go some way to filling the `holes' in these regions (see Figure
\ref{fig:holes}), due to the large transfer of dark matter that still occurs
(\S \ref{sec:haloindepdendent}), perhaps a different methodology is required.

Some of the holes that are present in lagrangain regions are vitally important;
these holes will collapse down to independent halos. An effort must be made
to ensure that those holes remain, whilst others are erased, with lower-mass
halos taking priority over their higher-mass cousins. With this hole-filling
exercise, it is important to note that even dark matter particles may now
have a different halo ID to lagrangain ID.

The methodology that is proposed here is as follows:
\begin{itemize}
    \item Initially define lagrangian regions in the same way as before for
          the dark matter
    \item Find the first $n$ neighbours of every dark matter particle
    \item Overwrite the lagrangian ID of these particles with the lowest (i.e.
          corresponding to the lowest mass halo) in the group
    \item Extend the lagrangian region definition to the gas particles in
          the same way as previously, by finding the closest gas neighbour
	  particle to every dark matter particle.
\end{itemize}


