\section{The \simba{} Simulation Suite}
\label{sec:simba}

\subsection{Code and sub-grid model}

This work uses the \simba{} simulation suite \citep{Dave2019}, which inherits
a large amount of physics from \mufasa{} \citep{Dave2016}. \simba{} uses a
variant of the GIZMO code \citep{Hopkins2015}, with the Meshless-Finite-Mass
(MFM) hydrodynamics solver using a cubic spline kernel with
64 neighbours. The gravitational forces are solved using the Tree-PM method
as described in \citet{Springel2005b} for Gadget-2, of which GIZMO is a
descendent. In the $50 \hmpc{}$, $512^3$ particle box used here, the mass
resolution for the gas elements is $1.7\times10^7h^{-1}$ M$_\odot$, and for
the dark matter is $7\times10^7h^{-1}$ M$_\odot$. The cosmology used in
\simba{} is consistent with results from \citet{PlanckCollaboration2016},
with $\Omega_\Lambda = 0.7$, $\Omega_{\rm m} = 0.3$, $\Omega_{\rm b} =
0.048$, $H_0 = 68$ km s$^{-1}$, $\sigma_8=0.82$, and $n_s=0.97$.

On top of this base code, the \simba{} sub-grid model is implemented. This
model is fully described in \citet{Dave2019}, but it is summarised here.
Radiative cooling and photoionisation are included from Grackle-3.1
\citep{Smith2016}. Stellar feedback is modelled using decoupled two-phase
winds that have 30\% of their ejected particles set at a temperature given by
the supernova energy minus the kinetic energy of the wind. The mass loading
factor of these winds scales with stellar mass using scalings from
\citet{AnglesAlcazar2017}, obtained from particle tracking in the FIRE
zoom-in simulations.

Black hole growth is included in \simba{} using the torque-limited accretion
model from \citet{AnglesAlcazar2017b} for cold gas and \citet{Bondi1952}
accretion for the hot gas. The AGN feedback model includes both kinetic winds
and X-ray feedback. At high Eddington ratios ($f_{\rm Edd} > 0.02$) or low
black holes mass ($M_{\rm BH} < 10^{7.5}$ M$_\odot$), the radiative-mode
winds are high mass-loaded and ejected at interstellar medium (ISM)
temperature with velocities $\lesssim 10^3 \kms{}$. At low Eddington ratios
and high black hole mass, the jet-mode winds are ejected at velocities
approaching $\sim 10^4\kms{}$. We refer the interested reader to the full
description of this feedback model in \citet{Dave2019}.

In addition to the fiducial model, we also use two comparison models. The
first, described as \nojet{}, includes all of the \simba{} physics but has
the high-energy black hole jet-mode winds disabled. All other star formation
and AGN feedback is included. The second, described as non-radiative, uses
the same initial conditions as the fiducial model but only includes
gravitational dynamics and hydrodynamics, i.e. without sub-grid models. This
latter simulation was performed with the {\sc Swift} simulation code
\citep{Schaller2016} using a Density-Entropy Smoothed Particle Hydrodynamics
(SPH) solver as it performs orders of magnitude faster than the original
GIZMO code \citep{Borrow2018}. The use of this hydrodynamics model, over the
MFM solver, will have a negligible effect on the quantities of interest in
this paper, as it has been shown that such a solver produces haloes of the
same baryonic mass when ran in non-radiative mode \citep[see
e.g.][]{Sembolini2016}.

\subsection{Defining haloes}

Haloes are defined using a modified version of the Amiga Halo Finder
\citep[AHF, ][]{Gill2004, Knollmann2009} presented in \citet{Muratov2015}.
This spherical overdensity finder determines the halo centers by using a
nested grid, and then fits parameters based on the Navarro-Frenk-White
\citep[NFW, ][]{Navarro1995} profile. Here we define the virial radius,
$R_{\rm vir}$, as the spherical overdensity radius retrieved from AHF consistent
with \citet{Bryan1998}. Substructure search was turned off, such that the
code only returned main haloes.

\subsection{Defining Lagrangian regions}

The Lagrangian region (LR) associated with a halo is the volume in the initial conditions
that contains the dark matter that will eventually collapse to form that halo.

Many methods exist for defining Lagrangian regions \citep[see e.g. ][ for a
collection of methods]{Onorbe2014}. In this work the Lagrangian regions are
defined in the following way:
\begin{enumerate}
	\item Find all haloes at redshift $z=0$, and assign them a unique halo ID.

    \item For each halo, match the particles contained within it with those
		  in the initial conditions. These particles are then assigned a Lagrangian
		  region ID that is the same as this halo ID, with particles outside of haloes
		  (and hence Lagrangian regions) assigned an ID of -1. This defines the initial
		  Lagrangian regions based on the dark matter.

	\item In some cases, discussed below, fill in the holes in this Lagrangian
		  region by using a nearest-neighbour search. In the fiducial case, skip
		  this step (see \S \ref{sec:convergence}).

	\item For every gas particle in the initial conditions, find the nearest dark
	      matter neighbour. This gas particle is assigned to the same Lagrangian
	      region as that dark matter particle.
\end{enumerate}
In this way, Lagrangian regions contain all dark matter particles that end up
within $R_{\rm vir}$ of each halo at $z=0$, by definition, as well as the
baryons that should also in principle collapse into the corresponding halo.
In \S \ref{sec:convergence}, we explore alternative definitions of LRs and
their impact in our results.