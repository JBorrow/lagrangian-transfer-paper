\section{The \simba{} Simulation Suite}
\label{sec:simba}

This work uses the \simba{} simulation suite \citep{Dave2019}, which inherits a
large amount of physics from \mufasa{} \citep{Dave2016}. \simba{} uses a
variant of the GIZMO code \citep{Hopkins2015}, with the Meshless-Finite-Mass
(MFM) hydrodynamics solver using a Wentland C2 kernel with 64 neighbours. The
gravitational forces are solved using the Tree-PM method as described in
\citet{Springel2005b} for Gadget-2, of which GIZMO is a descendent. In the $50 \hmpc{}$, $512^3$ particle box used here, the mass resolution for the gas elements is $1.7\times10^7h^{-1}$ M$_\odot$, and for the dark matter is $7\times10^7h^{-1}$ M$_\odot$. The cosmology used in \simba{} is consistent with results from
\citet{PlanckCollaboration2016}, with $\Omega_\Lambda = 0.7$, $\Omega_{\rm m} =
0.3$, $\Omega_{\rm b} = 0.048$, $H_0 = 68$ km s$^{-1}$, $\sigma_8=0.82$, and
$n_s=0.97$.

On top of this base code, the \simba{} sub-grid model is implemented. This
model is fully described in \citet{Dave2019}, but it is summarised here.
Radiative cooling and photoionisation are included from Grackle-3.1
\citep{Smith2016}. Stellar feedback is modelled using decoupled two-phase winds
that have 30\% of their ejected particles set at a temperature given by the
supernova energy minus the kinetic energy of the wind. The mass
loading factor of these winds scales with stellar mass using scalings from
\citet{AnglesAlcazar2017}, obtained from particle tracking in the FIRE zoom-in simulations.

Black hole growth and feedback is included in \simba{}, using the torque-limited
accretion model from \citet{AnglesAlcazar2017b} for cold gas and \citet{Bondi1952}
accretion for the hot gas. The AGN feedback model includes both kinetic
(winds) and energetic (X-ray) feedback. At high Eddington ratios or low black holes mass ($M_{\rm BH} < 10^{7.5}$
M$_\odot$), the
radiative-mode winds are ejected at interstellar medium (ISM) temperature at low velocity $\sim 10^3 \kms{}$. At low Eddington ratios and high black holes mass, the jet-mode winds are ejected at velocities approaching $\sim 10^4\kms{}$. We refer the interested reader to the full description of this
feedback model in \citet{Dave2019}.

In addition to the fiducial model, we also use two comparison models. The first, described as \nojet{}, includes all of the \simba{} physics, has the high-energy black hole jet-mode winds disabled.  All other star formation and AGN feedback is included.  The second, described as non-
radiative, uses the same initial conditions as the fiducial model but only
includes gravitational dynamics and hydrodynamics, i.e. without a sub-grid
model. This latter simulation was performed with the {\sc Swift} simulation
code \citep{Schaller2016} using a Density-Entropy Smoothed Particle
Hydrodynamics (SPH) solver as it performs orders of magnitude faster than the
original GIZMO code \citep{Borrow2018}. The use of this hydrodynamics model,
over the MFM solver, will have a negligible effect on the quantities of
interest in this paper (i.e. positions of particles at $z=0$), as it has been
shown that such a solver produces halos of the same baryonic mass when ran in
non-radiative mode \citep[see e.g.][]{Sembolini2016}.
