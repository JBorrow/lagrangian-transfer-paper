\section{Discussion and Conclusions}
\label{sec:conclusions}

We have developed two novel metrics that describe the movement of baryons
throughout a cosmological simulation with respect to the dark matter, and
employed them to investigate the \simba{} simulations and sub-grid model.
The first of these metrics, the {\it spread metric}, shows that:
\begin{itemize}
    \item Dark matter can be spread up to $7.5\hmpc{}$ away from their inital
          mass distribution throughout the course of a cosmological simulation.
          This has been validated with two simulation codes, \gizmo{} and \swift{}.
    \item Gas can be spread to even larger distances, with the distance
          dependent on the physics included in the sub-grid model. For the
          \simba{} galaxy formation model with AGN jets, we find that gas can
          be spread to up to $12\hmpc{}$ throughout the course of the
          simulation in a box that is only $50\hmpc{}$ in size, with 40\%
          (10\%) of baryons having moved $> 1\hmpc{}$ ($3\hmpc{}$). This is
          despite this powerful form of feedback only directly interacting
          with 0.4\% of particles, and points towards significant quantities
          of gas being entrained by these jets. It remains to be seen if this
          will increase further with higher mass objects in larger boxes.
    \item Stars in the simulation show a very similar level of spread to the
          dark matter, suggesting that the gas particles that stars form out
          of remain tightly coupled to the dark matter. This implies that the
          spreading of stars by gravitational dynamics dominates over the
          spreading of their gas particle progenitors by feedback.
    \item Using the spread metric to select particles, we have shown that
          dark matter that is spread to large distances forms the diffuse
          structure within and around haloes, with lower spread dark matter
          forming substructure within haloes. When extending this to the gas,
          we find that the baryons that are spread the most are those that
          reside in the diffuse structure around haloes, with this structure
          being created by the energetic feedback present in the \simba{}
          model. We suggest that this spread metric may be a useful, highly
          computationally efficient, way of selecting particles that have been
          entrained by feedback processes that are not tagged during the
          injection of energy.
\end{itemize}
The second of these metrics, which considers the baryonic make-up of haloes
at $z=0$ split by the Lagrangian origin of the particles, shows that:
\begin{itemize}
    \item Approximately 40\% of the gas in an average $z=0$ halo did not originate
          in the Lagrangian region of that halo, with around 30\% originating
          outside any Lagrangian region, and 10\% originating in the Lagrangian
          region of another halo. This suggests that \emph{inter-Lagrangian
          transfer} is prevalent throughout the simulation, with haloes interchanging
          particles between $z=2$ and $z=0$ thanks to energetic feedback pathways.
    \item The majority of the stellar component of haloes (90\% above a halo mass
          $10^{12}\msolar{}$) originates from the Lagrangian region of the
          same halo, as expected given the similar large-scale spreads of the
          stellar and dark matter.
    \item Below a halo mass of $10^{13}\msolar{}$, haloes can only retain
          approximately 20-30\% of the baryons from their Lagrangian region,
          with the majority of these baryons being lost to the IGM. Above
          this mass, haloes become strong enough gravitational wells to
          retain the majority of their baryons (up to 60\%) by around
          $10^{14}\msolar{}$ halo mass, although this result is somewhat
          uncertain due to the lack of objects in this mass range in the
          $50\hmpc{}$ simulation box used here.
    \item Haloes with mass $M_H > 10^{13.5}\msolar{}$, despite having a baryon
          fraction comparable to the cosmic mean, still show significant
          levels of transfer from other haloes and from outside any
          Lagrangian region. This suggests a complex cycling of baryons with
          approximately 20\% of their baryonic mass being `swapped' with the
          IGM by $z=0$.
    \item Different Lagrangian components, as they make up the baryon
          fraction of haloes, are affected differently by feedback mechanisms
          at different halo masses. In the halo mass range
          $10^{12}$--$10^{13}\msolar{}$, the component of baryonic mass from
          outside of the Lagrangian region is halved, whereas the component
          from the haloes own Lagrangian region is only reduced within 20\%;
          this highlights the importance of preventive feedback for the
          baryon fraction of haloes.
\end{itemize}

Our results add a new perspective to the connection between baryon cycling
and galaxy evolution. Using large volume simulations including
momentum-driven winds, \citet{Oppenheimer2010} showed that most stars likely
form out of gas that has previously been ejected in winds, and more recent
zoom-in simulations agree with the prevalence of wind recycling
\citep{Christensen2016, AnglesAlcazar2017, Tollet2019}. Using the FIRE
simulations, \citet{AnglesAlcazar2017} further showed that the intergalactic
transfer of gas between galaxies via winds can provide up to a third of the
stellar mass of Milky Way-mass galaxies. Here we have introduced the concept
of inter-Lagrangian transfer, which represents the extreme case of transfer
of baryons between individual central haloes. For the \simba{} simulations,
we find that only a small fraction (<5\%) of the stellar mass of haloes can
be made up from inter-Lagrangian transfer gas, suggesting that most
intergalactic transfer originates from satellite galaxies and is thus
confined within Lagrangian regions. It is nonetheless quite significant that
gas exchanged between Lagrangian regions can fuel star formation in a
different halo at all. In addition, we do find a significant contribution
(<20\%) of inter-Lagrangian transfer to the gas content of haloes at z = 0.
Recently, \citet{Hafen2019, Hafen2019b} has highlighted the contribution of
satellite winds to the gas and metal content of the CGM in the FIRE
simulations. Our results suggest that the origin of the CGM of galaxies is
linked to larger scales than previously considered.

These results provide two possible main implications for current works. The
first is the implications for semi-analytic models of galaxy formation. These
models, by construction, tie the baryonic matter to dark matter haloes; they
contain no prescription for gas that explicitly originates from regions where
the dark matter does not end the simulation in a bound object. Also, whilst
there has been some effort by \citet{Henriques2015, White2015} and others to
include wind recycling into these models, there is currently no semi-analytic
model that includes any concept of baryon transfer between un-merged haloes
or baryonic accretion rates significantly different to that expected from the
dark matter component.


The second implication is for zoom-in simulation suites. These suites
typically construct their initial conditions by considering the cubic volume,
ellipsoid, or convex hull in the initial conditions containing the dark
matter particles that are located within a given distance (typically
$2-3R_{\rm vir}$) of the selected halo at $z=0$ \citep[see e.g.][]{Onorbe2014}.
However, our results highlight that the shapes of the
causally connected regions in gas and dark matter may be significantly
different. For example, the Latte \citep{Wetzel2016} suite uses an exclusion
region for high resolution particles of around $1.5 \hmpc{}$ while we find
that 10\% of cosmological baryons can move >3 Mpc away relative to the
original neighbouring dark matter distribution. While zoom-in simulations are
constructed to avoid contamination of low-resolution particles into the
high-resolution region, our results suggest that they may miss a flux of
external baryons into the high resolution region. In practice, contamination
from external sources will be somewhat mitigated by the usual choice of
isolated haloes, but future work should consider these effects for zoom-in
suites that have a full hydrodynamical simulation for their parent.


The results presented here are based on the \simba{} model, which is in good
agreement with a wide range of galaxy \citep{Dave2019} and black hole
\citep{Thomas2019} observables, but are clearly dependent on the feedback
implementation. Other galaxy formation models may yield different results,
especially those with drastically different implementations for AGN feedback,
such as the purely thermal feedback in the EAGLE model
\citep{Schaye2015}. The spread metric represents a unique tool to
characterize the global effects of feedback and will enable novel comparisons
between existing cosmological simulations. Future work should also address
the connection between baryon spreading and galaxy/CGM observables, as well
as investigate baryonic effects on cosmological observables
\citep{Schneider2015, Chisari2018} in the context of the spread metric.
