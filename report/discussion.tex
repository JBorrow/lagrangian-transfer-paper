\section{Discussion}
\label{sec:conclusions}

In the above, we have developed two novel metrics that describe the different
ways in which dark matter and baryons flow in cosmological simulations. The
first, a simple distance metric, shows that it is not just the energetics of
the universe that are changed by the inclusion of AGN jets and other energetic
feedback mechanisms, but that the gas dynamics are changed completely. Whilst
different feedback models may be able to capture the same halo-level metrics
like the galaxy stellar mass function, they necessarily produce vastly
different dynamics. The second, which looks at transfer between lagrangian
regions as defined by the $z=0$ halo population, shows that $10\%$ of the
baryonic mass of given halo will have originated from the lagrangian region
of another halo. A common critique of \citet{AnglesAlcazar2017} is that the
majority of the objects that show transfer will have merged by $z=0$. Here,
for the first time, we showed that late-time transfer between halos occurs
in cosmological simulations.

We have also quantified the significant mis-match between the origins of
baryonic and dark matter in halos in cosmological simulations. $40\%$ of the
baryonic mass originates from a different spatial region in the initial
conditions to the dark matter contained in the halo.

We have shown that different lagrangian components, as they make up the
baryon fraction of halos, are affected differently by different feedback
mechanisms. Transfer from outside of any lagrangian region is predominantly
affected by (preventative) AGN feedback in the \simba{} model, whereas gas
that originated in the lagrangian region remains `locked-in' and only a small
proportion is evacuated by feedback.

The transfer between halos has been shown to be a late-time effect, with
there being a negligible proportion of the $z=2$ baryonic mass of halos
made up of particles from lagrangian regions defined by other halos.

A particularly concerning aspect to the baryon transfer results is that they
have the potential to have huge impacts on semi-analytic models of galaxy
formation. These models, by construction, tie the baryonic matter to dark
matter halos; they contain no prescription for gas that explicitly originates
from regions where the dark matter does not end the simulation in a bound
object. Also, whilst there has been some effort by \citet{Henriques2015} and
others to include wind recycling into these models, there is currently
no semi-analytic model that includes any concept of baryon transfer between
un-merged halos. The mixed origins of the baryons in the $z=0$ halos
point to a different physical origin for many fundamental galaxy properties.

