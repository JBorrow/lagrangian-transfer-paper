\section{Discussion and Conclusions}
\label{sec:conclusions}

We have developed two novel metrics that describe the movement of baryons
throughout a cosmological simulation with respect to the dark matter, and
employed them to investigate the \simba{} simulations and sub-grid model.
The first of these metrics, the {\it spread metric}, shows that:
\begin{itemize}
    \item Dark matter can be spread up to $7.5\hmpc{}$ away from its
          initial neighbour (of which the separation was only $80\hkpc{}$
          initially) throughout the course of a cosmological simulation. This
          has been validated with two simulation codes, \gizmo{} and \swift{}.
    \item Gas can be spread to even larger distances, with the distance
          dependent on the physics included in the sub-grid model. For the \simba{}
          galaxy formation model with AGN jets, we find that gas can be spread to
          up to $12\hmpc{}$ throughout the course of the simulation in a box that
          is only $50\hmpc{}$ large. This is despite this powerful form of feedback
          only directly interacting with 0.4\% of particles, and points towards
          significant quantities of gas being entrained by these jets. It remains
          to be seen if this will increase further with higher mass objects in
          larger boxes.
    \item Stars in the simulation show a very similar level of spread to the
          dark matter, suggesting that the gas particles that stars form out
          of remain tightly coupled to the dark matter.
    \item Using the spread metric to select particles, we have shown that
          dark matter that is spread to large distances forms the diffuse
          structure within and around haloes, with lower spread dark matter
          forming substructure within haloes. When extending this to the gas,
          we find that the baryons that are spread the most are those that
          reside in the diffuse structure around haloes, with this structure
          being created by the energetic feedback present in the \simba{}
          model. We suggest that this spread metric may be a useful, highly
          computationally efficient, way of selecting particles that have been
          entrained by feedback processes that are not tagged during the
          injection of energy.
\end{itemize}
The second of these metrics, which considers the baryonic make-up of haloes
at $z=0$ split by the Lagrangian origin of the particles, shows that:
\begin{itemize}
    \item Approximately 40\% of the gas in an average $z=0$ halo did not originate
          in the Lagrangian region of that halo, with around 30\% originating
          outside any Lagrangian region, and 10\% originating in the Lagrangian
          region of another halo. This suggests that \emph{inter-Lagrangian
          transfer} is prevalent throughout the simulation, with haloes interchanging
          particles between $z=2$ and $z=0$ thanks to energetic feedback pathways.
    \item The majority of the stellar component of haloes (90\% above a halo mass
          $10^{12}\msolar{}$) originates from the Lagrangian region of that halo,
          aligning with the results from the spread metric that these particles
          are very tightly coupled to the dark matter.
    \item Below a halo mass of $10^{13}\msolar{}$, haloes can only retain
          approximately 20\% of the baryons from their Lagrangian region,
          with the majority of these baryons being lost to the IGM. Above
          this mass, haloes become strong enough gravitational wells to retain
          the majority of their baryons by around $10^{14}\msolar{}$ halo mass,
          although this result is somewhat uncertain due to the $50\hmpc{}$ boxsize
          used here.
    \item These very large haloes, despite having a baryon fraction comparable
          to the cosmic mean, still show significant levels of transfer from other
          haloes and from outside any Lagrangian region. This suggests a complex
          cycling of baryons through these haloes, with approximately 20\% of their
          baryonic mass being `swapped' with the IGM by $z=0$.
    \item We have found that a very small fraction of the stellar mass of haloes
          can be made up from inter-Lagrangian transfer. Inter-galactic transfer,
          such as that presented in \citet{AnglesAlcazar2017}, that can provide
          up to a third of the stellar mass of the galaxy, appears to be confined
          within Lagrangian regions themselves. However, we do find significant levels
          ($\sim10\%$) of gas transfer between haloes by redshift $z=0$.
    \item Different Lagrangian components, as they make up the baryon
          fraction of haloes, are affected differently by feedback mechanisms at
          different halo masses. In the halo mass range
          $10^{12}$--$10^{13}\msolar{}$, the component of baryonic mass from
          outside of the Lagrangian region is halved, whereas the component from
          the haloes own Lagrangian region is only reduced within 20\%; this
          suggests that preventative feedback in late-stage formation could be more
          efficient at preventing baryon assembly in this mass range.
\end{itemize}
These results provide two possible main implications for current works. The
first is the implications for semi-analytic models of galaxy formation. These
models, by construction, tie the baryonic matter to dark matter haloes; they
contain no prescription for gas that explicitly originates from regions where
the dark matter does not end the simulation in a bound object. Also, whilst
there has been some effort by \citet{Henriques2015} and others to include
wind recycling into these models, there is currently no semi-analytic model
that includes any concept of baryon transfer between un-merged haloes. The
mixed origins of the baryons in the $z=0$ haloes point to a different
physical origin for many fundamental galaxy properties.

The second implication is for zoom-in simulation suites.These suites
typically construct their initial conditions by considering the cubic volume,
ellipsoid, or convex hull in the ICs containing the dark matter particles
that are located within a given distance of the selected halo at $z=0$
\citep[see e.g. ][]{Onorbe2014}. However, the above results highlight that
the shapes of the causally connected regions in gas and dark matter may be
significantly different. For example, the Latte \citep{Wetzel2016} suite uses
an exclusion region for high resolution particles of around $1.5 \hmpc{}$.
Whilst they do not find contamination of low-resolution particles into the
high-resolution region, the above metrics suggest that perhaps low resolution
\emph{gas} particles are not present due to a lack of sub-grid physics in the
unrefined region, preventing spread from the unrefined regions into the main
volume. This will be somewhat mitigated by the usual choice of very isolated
haloes for zoom-in simulations, but future work should focus on these
considerations for suites that have a full hydrodynamical simulation for
their parent.


Finally, we note that these metrics are only valid for the \simba{} model.
Other galaxy formation models may give different results, especially those
with drastically different implementations for AGN feedback such as the
purely thermal feedback in EAGLE. Future work should focus on the differences
between these models, as differences in their spread metrics or levels of
inter-Lagrangian transfer may be tied to differences in potential predicted
observables for these simulations, such as the metallicity distribution in
the CGM.