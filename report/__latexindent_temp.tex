@ARTICLE{AnglesAlcazar2017,
       author = {{Angl{\'e}s-Alc{\'a}zar}, Daniel and {Faucher-Gigu{\`e}re}, Claude-
        Andr{\'e} and {Kere{\v{s}}}, Du{\v{s}}an and {Hopkins}, Philip
        F. and {Quataert}, Eliot and {Murray}, Norman},
        title = "{The cosmic baryon cycle and galaxy mass assembly in the FIRE simulations}",
      journal = {\mnras},
     keywords = {galaxies: evolution, galaxies: formation, galaxies: star formation, intergalactic medium, cosmology: theory, Astrophysics - Astrophysics of Galaxies, Astrophysics - Cosmology and Nongalactic Astrophysics},
         year = 2017,
        month = Oct,
       volume = {470},
        pages = {4698-4719},
          doi = {10.1093/mnras/stx1517},
archivePrefix = {arXiv},
       eprint = {1610.08523},
 primaryClass = {astro-ph.GA},
       adsurl = {https://ui.adsabs.harvard.edu/\#abs/2017MNRAS.470.4698A},
      adsnote = {Provided by the SAO/NASA Astrophysics Data System}
}

@BOOK{Mo2010,
       author = {{Mo}, Houjun and {van den Bosch}, Frank C. and {White}, Simon},
        title = "{Galaxy Formation and Evolution}",
    booktitle = {Galaxy Formation and Evolution, by Houjun Mo , Frank van den Bosch , Simon White, Cambridge, UK: Cambridge University Press, 2010},
         year = 2010,
       adsurl = {https://ui.adsabs.harvard.edu/\#abs/2010gfe..book.....M},
      adsnote = {Provided by the SAO/NASA Astrophysics Data System}
}

@ARTICLE{Frenk1988,
       author = {{Frenk}, Carlos S. and {White}, Simon D.~M. and {Davis}, Marc and
        {Efstathiou}, George},
        title = "{The Formation of Dark Halos in a Universe Dominated by Cold Dark Matter}",
      journal = {\apj},
     keywords = {Cosmology, Dark Matter, Galactic Evolution, Galactic Structure, Spiral Galaxies, Stellar Luminosity, Astronomical Models, Computational Astrophysics, Galactic Clusters, Morphology, Universe, Astrophysics, COSMOLOGY, DARK MATTER, GALAXIES: FORMATION, GALAXIES: INTERNAL MOTIONS, GALAXIES: STRUCTURE, NUMERICAL METHODS},
         year = 1988,
        month = Apr,
       volume = {327},
        pages = {507},
          doi = {10.1086/166213},
       adsurl = {https://ui.adsabs.harvard.edu/\#abs/1988ApJ...327..507F},
      adsnote = {Provided by the SAO/NASA Astrophysics Data System}
}

@ARTICLE{Springel2005a,
       author = {{Springel}, Volker and {White}, Simon D.~M. and {Jenkins}, Adrian and
        {Frenk}, Carlos S. and {Yoshida}, Naoki and {Gao}, Liang and
        {Navarro}, Julio and {Thacker}, Robert and {Croton}, Darren and
        {Helly}, John and {Peacock}, John A. and {Cole}, Shaun and
        {Thomas}, Peter and {Couchman}, Hugh and {Evrard}, August and
        {Colberg}, J{\"o}rg and {Pearce}, Frazer},
        title = "{Simulations of the formation, evolution and clustering of galaxies and quasars}",
      journal = {\nat},
     keywords = {Astrophysics},
         year = 2005,
        month = Jun,
       volume = {435},
        pages = {629-636},
          doi = {10.1038/nature03597},
archivePrefix = {arXiv},
       eprint = {astro-ph/0504097},
 primaryClass = {astro-ph},
       adsurl = {https://ui.adsabs.harvard.edu/\#abs/2005Natur.435..629S},
      adsnote = {Provided by the SAO/NASA Astrophysics Data System}
}

@ARTICLE{Porter2014,
       author = {{Porter}, L.~A. and {Somerville}, R.~S. and {Primack}, J.~R. and
        {Croton}, D.~J. and {Covington}, M.~D. and {Graves}, G.~J. and
        {Faber}, S.~M.},
        title = "{Modelling the ages and metallicities of early-type galaxies in Fundamental Plane space}",
      journal = {\mnras},
     keywords = {galaxies: elliptical and lenticular, cD, galaxies: evolution, galaxies: formation, galaxies: interactions, Astrophysics - Astrophysics of Galaxies},
         year = 2014,
        month = Dec,
       volume = {445},
        pages = {3092-3104},
          doi = {10.1093/mnras/stu1701},
archivePrefix = {arXiv},
       eprint = {1407.2186},
 primaryClass = {astro-ph.GA},
       adsurl = {https://ui.adsabs.harvard.edu/\#abs/2014MNRAS.445.3092P},
      adsnote = {Provided by the SAO/NASA Astrophysics Data System}
}

@ARTICLE{Henriques2015,
       author = {{Henriques}, Bruno M.~B. and {White}, Simon D.~M. and {Thomas}, Peter A.
        and {Angulo}, Raul and {Guo}, Qi and {Lemson}, Gerard and
        {Springel}, Volker and {Overzier}, Roderik},
        title = "{Galaxy formation in the Planck cosmology - I. Matching the observed evolution of star formation rates, colours and stellar masses}",
      journal = {\mnras},
     keywords = {methods: analytical, methods: statistical, galaxies: evolution, galaxies: formation, galaxies: high-redshift, Astrophysics - Astrophysics of Galaxies, Astrophysics - Cosmology and Nongalactic Astrophysics},
         year = 2015,
        month = Aug,
       volume = {451},
        pages = {2663-2680},
          doi = {10.1093/mnras/stv705},
archivePrefix = {arXiv},
       eprint = {1410.0365},
 primaryClass = {astro-ph.GA},
       adsurl = {https://ui.adsabs.harvard.edu/\#abs/2015MNRAS.451.2663H},
      adsnote = {Provided by the SAO/NASA Astrophysics Data System}
}

@ARTICLE{Lacey2016,
       author = {{Lacey}, Cedric G. and {Baugh}, Carlton M. and {Frenk}, Carlos S. and
        {Benson}, Andrew J. and {Bower}, Richard G. and {Cole}, Shaun
        and {Gonzalez-Perez}, Violeta and {Helly}, John C. and {Lagos},
        Claudia D.~P. and {Mitchell}, Peter D.},
        title = "{A unified multiwavelength model of galaxy formation}",
      journal = {\mnras},
     keywords = {galaxies: evolution, galaxies: formation, galaxies: high-redshift, Astrophysics - Astrophysics of Galaxies},
         year = 2016,
        month = Nov,
       volume = {462},
        pages = {3854-3911},
          doi = {10.1093/mnras/stw1888},
archivePrefix = {arXiv},
       eprint = {1509.08473},
 primaryClass = {astro-ph.GA},
       adsurl = {https://ui.adsabs.harvard.edu/\#abs/2016MNRAS.462.3854L},
      adsnote = {Provided by the SAO/NASA Astrophysics Data System}
}

@ARTICLE{Hernquist1989,
       author = {{Hernquist}, Lars and {Katz}, Neal},
        title = "{TREESPH: A Unification of SPH with the Hierarchical Tree Method}",
      journal = {The Astrophysical Journal Supplement Series},
     keywords = {Computational Fluid Dynamics, Computerized Simulation, Data Smoothing, Magnetohydrodynamics, Trees (Mathematics), Dynamical Systems, Many Body Problem, Monte Carlo Method, Spatial Resolution, Fluid Mechanics and Heat Transfer, HYDRODYNAMICS, NUMERICAL METHODS},
         year = 1989,
        month = Jun,
       volume = {70},
        pages = {419},
          doi = {10.1086/191344},
       adsurl = {https://ui.adsabs.harvard.edu/\#abs/1989ApJS...70..419H},
      adsnote = {Provided by the SAO/NASA Astrophysics Data System}
}

@ARTICLE{Teyssier2002,
       author = {{Teyssier}, R.},
        title = "{Cosmological hydrodynamics with adaptive mesh refinement. A new high resolution code called RAMSES}",
      journal = {\aap},
     keywords = {GRAVITATION, HYDRODYNAMICS, METHODS: NUMERICAL, COSMOLOGY: THEORY, COSMOLOGY: LARGE-SCALE STRUCTURE OF UNIVERSE, Astrophysics},
         year = 2002,
        month = Apr,
       volume = {385},
        pages = {337-364},
          doi = {10.1051/0004-6361:20011817},
archivePrefix = {arXiv},
       eprint = {astro-ph/0111367},
 primaryClass = {astro-ph},
       adsurl = {https://ui.adsabs.harvard.edu/\#abs/2002A&A...385..337T},
      adsnote = {Provided by the SAO/NASA Astrophysics Data System}
}

@ARTICLE{Springel2005b,
       author = {{Springel}, Volker},
        title = "{The cosmological simulation code GADGET-2}",
      journal = {\mnras},
     keywords = {methods: numerical, galaxies: interactions, dark matter, Astrophysics},
         year = 2005,
        month = Dec,
       volume = {364},
        pages = {1105-1134},
          doi = {10.1111/j.1365-2966.2005.09655.x},
archivePrefix = {arXiv},
       eprint = {astro-ph/0505010},
 primaryClass = {astro-ph},
       adsurl = {https://ui.adsabs.harvard.edu/\#abs/2005MNRAS.364.1105S},
      adsnote = {Provided by the SAO/NASA Astrophysics Data System}
}

@ARTICLE{Revaz2012,
       author = {{Revaz}, Y. and {Jablonka}, P.},
        title = "{The dynamical and chemical evolution of dwarf spheroidal galaxies with GEAR}",
      journal = {\aap},
     keywords = {galaxies: evolution, dark matter, galaxies: dwarf, galaxies: abundances, galaxies: formation, galaxies: star formation, Astrophysics - Cosmology and Nongalactic Astrophysics, Astrophysics - Astrophysics of Galaxies},
         year = 2012,
        month = Feb,
       volume = {538},
          eid = {A82},
        pages = {A82},
          doi = {10.1051/0004-6361/201117402},
archivePrefix = {arXiv},
       eprint = {1109.0989},
 primaryClass = {astro-ph.CO},
       adsurl = {https://ui.adsabs.harvard.edu/\#abs/2012A&A...538A..82R},
      adsnote = {Provided by the SAO/NASA Astrophysics Data System}
}

@ARTICLE{Vogelsberger2014,
       author = {{Vogelsberger}, Mark and {Genel}, Shy and {Springel}, Volker and
        {Torrey}, Paul and {Sijacki}, Debora and {Xu}, Dandan and
        {Snyder}, Greg and {Nelson}, Dylan and {Hernquist}, Lars},
        title = "{Introducing the Illustris Project: simulating the coevolution of dark and visible matter in the Universe}",
      journal = {\mnras},
     keywords = {methods: numerical, cosmology: theory, Astrophysics - Cosmology and Nongalactic Astrophysics},
         year = 2014,
        month = Oct,
       volume = {444},
        pages = {1518-1547},
          doi = {10.1093/mnras/stu1536},
archivePrefix = {arXiv},
       eprint = {1405.2921},
 primaryClass = {astro-ph.CO},
       adsurl = {https://ui.adsabs.harvard.edu/\#abs/2014MNRAS.444.1518V},
      adsnote = {Provided by the SAO/NASA Astrophysics Data System}
}

@ARTICLE{Schaye2015,
       author = {{Schaye}, Joop and {Crain}, Robert A. and {Bower}, Richard G. and
        {Furlong}, Michelle and {Schaller}, Matthieu and {Theuns}, Tom
        and {Dalla Vecchia}, Claudio and {Frenk}, Carlos S. and
        {McCarthy}, I.~G. and {Helly}, John C. and {Jenkins}, Adrian and
        {Rosas-Guevara}, Y.~M. and {White}, Simon D.~M. and {Baes},
        Maarten and {Booth}, C.~M. and {Camps}, Peter and {Navarro},
        Julio F. and {Qu}, Yan and {Rahmati}, Alireza and {Sawala}, Till
        and {Thomas}, Peter A. and {Trayford}, James},
        title = "{The EAGLE project: simulating the evolution and assembly of galaxies and their environments}",
      journal = {\mnras},
     keywords = {methods: numerical, galaxies: evolution, galaxies: formation, cosmology: theory, Astrophysics - Astrophysics of Galaxies, Astrophysics - Cosmology and Nongalactic Astrophysics},
         year = 2015,
        month = Jan,
       volume = {446},
        pages = {521-554},
          doi = {10.1093/mnras/stu2058},
archivePrefix = {arXiv},
       eprint = {1407.7040},
 primaryClass = {astro-ph.GA},
       adsurl = {https://ui.adsabs.harvard.edu/\#abs/2015MNRAS.446..521S},
      adsnote = {Provided by the SAO/NASA Astrophysics Data System}
}

@ARTICLE{Hopkins2014,
       author = {{Hopkins}, Philip F. and {Kere{\v{s}}}, Du{\v{s}}an and {O{\~n}orbe},
        Jos{\'e} and {Faucher-Gigu{\`e}re}, Claude-Andr{\'e} and
        {Quataert}, Eliot and {Murray}, Norman and {Bullock}, James S.},
        title = "{Galaxies on FIRE (Feedback In Realistic Environments): stellar feedback explains cosmologically inefficient star formation}",
      journal = {\mnras},
     keywords = {stars: formation, galaxies: active, galaxies: evolution, galaxies: formation, cosmology: theory, Astrophysics - Cosmology and Nongalactic Astrophysics, Astrophysics - Astrophysics of Galaxies},
         year = 2014,
        month = Nov,
       volume = {445},
        pages = {581-603},
          doi = {10.1093/mnras/stu1738},
archivePrefix = {arXiv},
       eprint = {1311.2073},
 primaryClass = {astro-ph.CO},
       adsurl = {https://ui.adsabs.harvard.edu/\#abs/2014MNRAS.445..581H},
      adsnote = {Provided by the SAO/NASA Astrophysics Data System}
}

@ARTICLE{Hopkins2018,
   author = {{Hopkins}, P.~F. and {Wetzel}, A. and {Kere{\v s}}, D. and {Faucher-Gigu{\`e}re}, C.-A. and 
	{Quataert}, E. and {Boylan-Kolchin}, M. and {Murray}, N. and 
	{Hayward}, C.~C. and {Garrison-Kimmel}, S. and {Hummels}, C. and 
	{Feldmann}, R. and {Torrey}, P. and {Ma}, X. and {Angl{\'e}s-Alc{\'a}zar}, D. and 
	{Su}, K.-Y. and {Orr}, M. and {Schmitz}, D. and {Escala}, I. and 
	{Sanderson}, R. and {Grudi{\'c}}, M.~Y. and {Hafen}, Z. and 
	{Kim}, J.-H. and {Fitts}, A. and {Bullock}, J.~S. and {Wheeler}, C. and 
	{Chan}, T.~K. and {Elbert}, O.~D. and {Narayanan}, D.},
    title = "{FIRE-2 simulations: physics versus numerics in galaxy formation}",
  journal = {\mnras},
archivePrefix = "arXiv",
   eprint = {1702.06148},
 keywords = {methods: numerical, stars: formation, galaxies: active, galaxies: evolution, galaxies: formation, cosmology: theory},
     year = 2018,
    month = oct,
   volume = 480,
    pages = {800-863},
      doi = {10.1093/mnras/sty1690},
   adsurl = {http://adsabs.harvard.edu/abs/2018MNRAS.480..800H},
  adsnote = {Provided by the SAO/NASA Astrophysics Data System}
}

@ARTICLE{BenitezLlambay2018,
       author = {{Ben{\'\i}tez-Llambay}, Alejandro and {Navarro}, Julio F. and {Frenk},
        Carlos S. and {Ludlow}, Aaron D.},
        title = "{The vertical structure of gaseous galaxy discs in cold dark matter haloes}",
      journal = {\mnras},
     keywords = {galaxies: formation, galaxies: fundamental parameters, galaxies: haloes, galaxies: structure, Astrophysics - Astrophysics of Galaxies},
         year = 2018,
        month = Jan,
       volume = {473},
        pages = {1019-1037},
          doi = {10.1093/mnras/stx2420},
archivePrefix = {arXiv},
       eprint = {1707.08046},
 primaryClass = {astro-ph.GA},
       adsurl = {https://ui.adsabs.harvard.edu/\#abs/2018MNRAS.473.1019B},
      adsnote = {Provided by the SAO/NASA Astrophysics Data System}
}

@ARTICLE{Wetzel2016,
       author = {{Wetzel}, Andrew R. and {Hopkins}, Philip F. and {Kim}, Ji-hoon and
        {Faucher-Gigu{\`e}re}, Claude-Andr{\'e} and {Kere{\v{s}}},
        Du{\v{s}}an and {Quataert}, Eliot},
        title = "{Reconciling Dwarf Galaxies with {\ensuremath{\Lambda}}CDM Cosmology: Simulating a Realistic Population of Satellites around a Milky Way-mass Galaxy}",
      journal = {\apj},
     keywords = {cosmology: theory, galaxies: dwarf, galaxies: formation, galaxies: star formation, Local Group, methods: numerical, Astrophysics - Astrophysics of Galaxies},
         year = 2016,
        month = Aug,
       volume = {827},
          eid = {L23},
        pages = {L23},
          doi = {10.3847/2041-8205/827/2/L23},
archivePrefix = {arXiv},
       eprint = {1602.05957},
 primaryClass = {astro-ph.GA},
       adsurl = {https://ui.adsabs.harvard.edu/\#abs/2016ApJ...827L..23W},
      adsnote = {Provided by the SAO/NASA Astrophysics Data System}
}

@ARTICLE{Dave2016,
       author = {{Dav{\'e}}, Romeel and {Thompson}, Robert and {Hopkins}, Philip F.},
        title = "{MUFASA: galaxy formation simulations with meshless hydrodynamics}",
      journal = {\mnras},
     keywords = {galaxies: evolution, galaxies: formation, Astrophysics - Astrophysics of Galaxies},
         year = 2016,
        month = Nov,
       volume = {462},
        pages = {3265-3284},
          doi = {10.1093/mnras/stw1862},
archivePrefix = {arXiv},
       eprint = {1604.01418},
 primaryClass = {astro-ph.GA},
       adsurl = {https://ui.adsabs.harvard.edu/\#abs/2016MNRAS.462.3265D},
      adsnote = {Provided by the SAO/NASA Astrophysics Data System}
}

@ARTICLE{Hopkins2015,
       author = {{Hopkins}, Philip F.},
        title = "{A new class of accurate, mesh-free hydrodynamic simulation methods}",
      journal = {\mnras},
     keywords = {hydrodynamics, instabilities, turbulence, methods: numerical, cosmology: theory, Astrophysics - Cosmology and Nongalactic Astrophysics, Astrophysics - Astrophysics of Galaxies, Astrophysics - Instrumentation and Methods for Astrophysics, Physics - Computational Physics, Physics - Fluid Dynamics},
         year = 2015,
        month = Jun,
       volume = {450},
        pages = {53-110},
          doi = {10.1093/mnras/stv195},
archivePrefix = {arXiv},
       eprint = {1409.7395},
 primaryClass = {astro-ph.CO},
       adsurl = {https://ui.adsabs.harvard.edu/\#abs/2015MNRAS.450...53H},
      adsnote = {Provided by the SAO/NASA Astrophysics Data System}
}

@ARTICLE{PlanckCollaboration2016,
       author = {{Planck Collaboration} and {Ade}, P.~A.~R. and {Aghanim}, N. and
        {Arnaud}, M. and {Ashdown}, M. and {Aumont}, J. and
        {Baccigalupi}, C. and {Banday}, A.~J. and {Barreiro}, R.~B. and
        {Bartlett}, J.~G. and {Bartolo}, N. and {Battaner}, E. and
        {Battye}, R. and {Benabed}, K. and {Beno{\^\i}t}, A. and
        {Benoit-L{\'e}vy}, A. and {Bernard}, J. -P. and {Bersanelli}, M.
        and {Bielewicz}, P. and {Bock}, J.~J. and {Bonaldi}, A. and
        {Bonavera}, L. and {Bond}, J.~R. and {Borrill}, J. and
        {Bouchet}, F.~R. and {Boulanger}, F. and {Bucher}, M. and
        {Burigana}, C. and {Butler}, R.~C. and {Calabrese}, E. and
        {Cardoso}, J. -F. and {Catalano}, A. and {Challinor}, A. and
        {Chamballu}, A. and {Chary}, R. -R. and {Chiang}, H.~C. and
        {Chluba}, J. and {Christensen}, P.~R. and {Church}, S. and
        {Clements}, D.~L. and {Colombi}, S. and {Colombo}, L.~P.~L. and
        {Combet}, C. and {Coulais}, A. and {Crill}, B.~P. and {Curto},
        A. and {Cuttaia}, F. and {Danese}, L. and {Davies}, R.~D. and
        {Davis}, R.~J. and {de Bernardis}, P. and {de Rosa}, A. and {de
        Zotti}, G. and {Delabrouille}, J. and {D{\'e}sert}, F. -X. and
        {Di Valentino}, E. and {Dickinson}, C. and {Diego}, J.~M. and
        {Dolag}, K. and {Dole}, H. and {Donzelli}, S. and {Dor{\'e}}, O.
        and {Douspis}, M. and {Ducout}, A. and {Dunkley}, J. and
        {Dupac}, X. and {Efstathiou}, G. and {Elsner}, F. and
        {En{\ss}lin}, T.~A. and {Eriksen}, H.~K. and {Farhang}, M. and
        {Fergusson}, J. and {Finelli}, F. and {Forni}, O. and {Frailis},
        M. and {Fraisse}, A.~A. and {Franceschi}, E. and {Frejsel}, A.
        and {Galeotta}, S. and {Galli}, S. and {Ganga}, K. and
        {Gauthier}, C. and {Gerbino}, M. and {Ghosh}, T. and {Giard}, M.
        and {Giraud-H{\'e}raud}, Y. and {Giusarma}, E. and {Gjerl{\o}w},
        E. and {Gonz{\'a}lez-Nuevo}, J. and {G{\'o}rski}, K.~M. and
        {Gratton}, S. and {Gregorio}, A. and {Gruppuso}, A. and
        {Gudmundsson}, J.~E. and {Hamann}, J. and {Hansen}, F.~K. and
        {Hanson}, D. and {Harrison}, D.~L. and {Helou}, G. and {Henrot-
        Versill{\'e}}, S. and {Hern{\'a}ndez-Monteagudo}, C. and
        {Herranz}, D. and {Hildebrandt}, S.~R. and {Hivon}, E. and
        {Hobson}, M. and {Holmes}, W.~A. and {Hornstrup}, A. and
        {Hovest}, W. and {Huang}, Z. and {Huffenberger}, K.~M. and
        {Hurier}, G. and {Jaffe}, A.~H. and {Jaffe}, T.~R. and {Jones},
        W.~C. and {Juvela}, M. and {Keih{\"a}nen}, E. and {Keskitalo},
        R. and {Kisner}, T.~S. and {Kneissl}, R. and {Knoche}, J. and
        {Knox}, L. and {Kunz}, M. and {Kurki-Suonio}, H. and {Lagache},
        G. and {L{\"a}hteenm{\"a}ki}, A. and {Lamarre}, J. -M. and
        {Lasenby}, A. and {Lattanzi}, M. and {Lawrence}, C.~R. and
        {Leahy}, J.~P. and {Leonardi}, R. and {Lesgourgues}, J. and
        {Levrier}, F. and {Lewis}, A. and {Liguori}, M. and {Lilje},
        P.~B. and {Linden-V{\o}rnle}, M. and {L{\'o}pez-Caniego}, M. and
        {Lubin}, P.~M. and {Mac{\'\i}as-P{\'e}rez}, J.~F. and {Maggio},
        G. and {Maino}, D. and {Mandolesi}, N. and {Mangilli}, A. and
        {Marchini}, A. and {Maris}, M. and {Martin}, P.~G. and
        {Martinelli}, M. and {Mart{\'\i}nez-Gonz{\'a}lez}, E. and
        {Masi}, S. and {Matarrese}, S. and {McGehee}, P. and {Meinhold},
        P.~R. and {Melchiorri}, A. and {Melin}, J. -B. and {Mendes}, L.
        and {Mennella}, A. and {Migliaccio}, M. and {Millea}, M. and
        {Mitra}, S. and {Miville-Desch{\^e}nes}, M. -A. and {Moneti}, A.
        and {Montier}, L. and {Morgante}, G. and {Mortlock}, D. and
        {Moss}, A. and {Munshi}, D. and {Murphy}, J.~A. and {Naselsky},
        P. and {Nati}, F. and {Natoli}, P. and {Netterfield}, C.~B. and
        {N{\o}rgaard-Nielsen}, H.~U. and {Noviello}, F. and {Novikov},
        D. and {Novikov}, I. and {Oxborrow}, C.~A. and {Paci}, F. and
        {Pagano}, L. and {Pajot}, F. and {Paladini}, R. and {Paoletti},
        D. and {Partridge}, B. and {Pasian}, F. and {Patanchon}, G. and
        {Pearson}, T.~J. and {Perdereau}, O. and {Perotto}, L. and
        {Perrotta}, F. and {Pettorino}, V. and {Piacentini}, F. and
        {Piat}, M. and {Pierpaoli}, E. and {Pietrobon}, D. and
        {Plaszczynski}, S. and {Pointecouteau}, E. and {Polenta}, G. and
        {Popa}, L. and {Pratt}, G.~W. and {Pr{\'e}zeau}, G. and
        {Prunet}, S. and {Puget}, J. -L. and {Rachen}, J.~P. and
        {Reach}, W.~T. and {Rebolo}, R. and {Reinecke}, M. and
        {Remazeilles}, M. and {Renault}, C. and {Renzi}, A. and
        {Ristorcelli}, I. and {Rocha}, G. and {Rosset}, C. and
        {Rossetti}, M. and {Roudier}, G. and {Rouill{\'e} d'Orfeuil}, B.
        and {Rowan-Robinson}, M. and {Rubi{\~n}o-Mart{\'\i}n}, J.~A. and
        {Rusholme}, B. and {Said}, N. and {Salvatelli}, V. and
        {Salvati}, L. and {Sandri}, M. and {Santos}, D. and
        {Savelainen}, M. and {Savini}, G. and {Scott}, D. and
        {Seiffert}, M.~D. and {Serra}, P. and {Shellard}, E.~P.~S. and
        {Spencer}, L.~D. and {Spinelli}, M. and {Stolyarov}, V. and
        {Stompor}, R. and {Sudiwala}, R. and {Sunyaev}, R. and {Sutton},
        D. and {Suur-Uski}, A. -S. and {Sygnet}, J. -F. and {Tauber},
        J.~A. and {Terenzi}, L. and {Toffolatti}, L. and {Tomasi}, M.
        and {Tristram}, M. and {Trombetti}, T. and {Tucci}, M. and
        {Tuovinen}, J. and {T{\"u}rler}, M. and {Umana}, G. and
        {Valenziano}, L. and {Valiviita}, J. and {Van Tent}, F. and
        {Vielva}, P. and {Villa}, F. and {Wade}, L.~A. and {Wandelt},
        B.~D. and {Wehus}, I.~K. and {White}, M. and {White}, S.~D.~M.
        and {Wilkinson}, A. and {Yvon}, D. and {Zacchei}, A. and
        {Zonca}, A.},
        title = "{Planck 2015 results. XIII. Cosmological parameters}",
      journal = {\aap},
     keywords = {cosmology: observations, cosmology: theory, cosmic background radiation, cosmological parameters, Astrophysics - Cosmology and Nongalactic Astrophysics},
         year = 2016,
        month = Sep,
       volume = {594},
          eid = {A13},
        pages = {A13},
          doi = {10.1051/0004-6361/201525830},
archivePrefix = {arXiv},
       eprint = {1502.01589},
 primaryClass = {astro-ph.CO},
       adsurl = {https://ui.adsabs.harvard.edu/\#abs/2016A&A...594A..13P},
      adsnote = {Provided by the SAO/NASA Astrophysics Data System}
}

@MISC{Smith2016,
       author = {{Smith}, Britton D. and {Bryan}, Greg L. and {Glover}, Simon C.~O. and
        {Goldbaum}, Nathan J. and {Turk}, Matthew J. and {Regan}, John
        and {Wise}, John H. and {Schive}, Hsi-Yu and {Abel}, Tom and
        {Emerick}, Andrew and {O'Shea}, Brian W. and {Anninos}, Peter
        and {Hummels}, Cameron B. and {Khochfar}, Sadegh},
        title = "{Grackle: Chemistry and radiative cooling library for astrophysical simulations}",
     keywords = {Software},
         year = 2016,
        month = Dec,
          eid = {ascl:1612.020},
        pages = {ascl:1612.020},
archivePrefix = {ascl},
       eprint = {1612.020},
       adsurl = {https://ui.adsabs.harvard.edu/\#abs/2016ascl.soft12020S},
      adsnote = {Provided by the SAO/NASA Astrophysics Data System}
}

@ARTICLE{Muratov2015,
       author = {{Muratov}, Alexander L. and {Kere{\v{s}}}, Du{\v{s}}an and {Faucher-
        Gigu{\`e}re}, Claude-Andr{\'e} and {Hopkins}, Philip F. and
        {Quataert}, Eliot and {Murray}, Norman},
        title = "{Gusty, gaseous flows of FIRE: galactic winds in cosmological simulations with explicit stellar feedback}",
      journal = {\mnras},
     keywords = {stars: formation, galaxies: evolution, galaxies: formation, cosmology: theory, Astrophysics - Astrophysics of Galaxies},
         year = 2015,
        month = Dec,
       volume = {454},
        pages = {2691-2713},
          doi = {10.1093/mnras/stv2126},
archivePrefix = {arXiv},
       eprint = {1501.03155},
 primaryClass = {astro-ph.GA},
       adsurl = {https://ui.adsabs.harvard.edu/\#abs/2015MNRAS.454.2691M},
      adsnote = {Provided by the SAO/NASA Astrophysics Data System}
}

@ARTICLE{AnglesAlcazar2017b,
       author = {{Angl{\'e}s-Alc{\'a}zar}, Daniel and {Dav{\'e}}, Romeel and {Faucher-
        Gigu{\`e}re}, Claude-Andr{\'e} and {{\"O}zel}, Feryal and
        {Hopkins}, Philip F.},
        title = "{Gravitational torque-driven black hole growth and feedback in cosmological simulations}",
      journal = {\mnras},
     keywords = {galaxies: active, galaxies: evolution, galaxies: formation, intergalactic medium, quasars: supermassive black holes, cosmology: theory, Astrophysics - Astrophysics of Galaxies, Astrophysics - Cosmology and Nongalactic Astrophysics, Astrophysics - High Energy Astrophysical Phenomena},
         year = 2017,
        month = Jan,
       volume = {464},
        pages = {2840-2853},
          doi = {10.1093/mnras/stw2565},
archivePrefix = {arXiv},
       eprint = {1603.08007},
 primaryClass = {astro-ph.GA},
       adsurl = {https://ui.adsabs.harvard.edu/\#abs/2017MNRAS.464.2840A},
      adsnote = {Provided by the SAO/NASA Astrophysics Data System}
}

@ARTICLE{Bondi1952,
       author = {{Bondi}, H.},
        title = "{On spherically symmetrical accretion}",
      journal = {\mnras},
         year = 1952,
        month = Jan,
       volume = {112},
        pages = {195},
          doi = {10.1093/mnras/112.2.195},
       adsurl = {https://ui.adsabs.harvard.edu/\#abs/1952MNRAS.112..195B},
      adsnote = {Provided by the SAO/NASA Astrophysics Data System}
}

@ARTICLE{Dave2019,
       author = {{Dav{\'e}}, Romeel and {Angl{\'e}s-Alc{\'a}zar}, Daniel and
         {Narayanan}, Desika and {Li}, Qi and {Rafieferantsoa}, Mika H. and
         {Appleby}, Sarah},
        title = "{Simba: Cosmological Simulations with Black Hole Growth and Feedback}",
      journal = {arXiv e-prints},
     keywords = {Astrophysics - Astrophysics of Galaxies, Astrophysics - Cosmology and Nongalactic Astrophysics},
         year = "2019",
        month = "Jan",
          eid = {arXiv:1901.10203},
        pages = {arXiv:1901.10203},
archivePrefix = {arXiv},
       eprint = {1901.10203},
 primaryClass = {astro-ph.GA},
       adsurl = {https://ui.adsabs.harvard.edu/\#abs/2019arXiv190110203D},
      adsnote = {Provided by the SAO/NASA Astrophysics Data System}
}

@ARTICLE{Sembolini2016,
       author = {{Sembolini}, Federico and {Yepes}, Gustavo and {Pearce}, Frazer R. and
         {Knebe}, Alexander and {Kay}, Scott T. and {Power}, Chris and
         {Cui}, Weiguang and {Beck}, Alexander M. and {Borgani}, Stefano and
         {Dalla Vecchia}, Claudio and {Dav{\'e}}, Romeel and
         {Elahi}, Pascal Jahan and {February}, Sean and {Huang}, Shuiyao and
         {Hobbs}, Alex and {Katz}, Neal and {Lau}, Erwin and {McCarthy}, Ian G. and
         {Murante}, Guiseppe and {Nagai}, Daisuke and {Nelson}, Kaylea and
         {Newton}, Richard D.~A. and {Perret}, Valentin and {Puchwein}, Ewald and
         {Read}, Justin I. and {Saro}, Alexandro and {Schaye}, Joop and
         {Teyssier}, Romain and {Thacker}, Robert J.},
        title = "{nIFTy galaxy cluster simulations - I. Dark matter and non-radiative models}",
      journal = {\mnras},
     keywords = {methods: numerical, galaxies: haloes, cosmology: theory, dark matter, Astrophysics - Cosmology and Nongalactic Astrophysics},
         year = "2016",
        month = "Apr",
       volume = {457},
        pages = {4063-4080},
          doi = {10.1093/mnras/stw250},
archivePrefix = {arXiv},
       eprint = {1503.06065},
 primaryClass = {astro-ph.CO},
       adsurl = {https://ui.adsabs.harvard.edu/\#abs/2016MNRAS.457.4063S},
      adsnote = {Provided by the SAO/NASA Astrophysics Data System}
}

@ARTICLE{Borrow2018,
       author = {{Borrow}, Josh and {Bower}, Richard G. and {Draper}, Peter W. and
         {Gonnet}, Pedro and {Schaller}, Matthieu},
        title = "{SWIFT: Maintaining weak-scalability with a dynamic range of \$10\^4\$ in time-step size to harness extreme adaptivity}",
      journal = "{Proceedings of the 13th SPHERIC International Workshop, Galway, Ireland, June 26-28 2018}",
     keywords = {Computer Science - Distributed, Parallel, and Cluster Computing, Astrophysics - Instrumentation and Methods for Astrophysics, Computer Science - Data Structures and Algorithms},
         year = "2018",
        month = "Jul",
          eid = {arXiv:1807.01341},
        pages = {44-51},
archivePrefix = {arXiv},
       eprint = {1807.01341},
 primaryClass = {cs.DC},
       adsurl = {https://ui.adsabs.harvard.edu/\#abs/2018arXiv180701341B},
      adsnote = {Provided by the SAO/NASA Astrophysics Data System}
}

@ARTICLE{Schaller2016,
       author = {{Schaller}, Matthieu and {Gonnet}, Pedro and {Chalk}, Aidan B.~G. and
         {Draper}, Peter W.},
        title = "{SWIFT: Using task-based parallelism, fully asynchronous communication, and graph partition-based domain decomposition for strong scaling on more than 100,000 cores}",
      journal = {Proceedings of the Platform for Advanced Scientific Computing Conference, 2016},
     keywords = {Computer Science - Distributed, Parallel, and Cluster Computing, Astrophysics - Instrumentation and Methods for Astrophysics},
         year = "2016",
        month = "Jun",
          eid = {arXiv:1606.02738},
        pages = {1-10},
        address= "Lausanne, Switzerland",
        volume= "2",
archivePrefix = {arXiv},
       eprint = {1606.02738},
 primaryClass = {cs.DC},
       adsurl = {https://ui.adsabs.harvard.edu/\#abs/2016arXiv160602738S},
      adsnote = {Provided by the SAO/NASA Astrophysics Data System}
}

@ARTICLE{Elahi2019,
       author = {{Elahi}, Pascal J. and {Ca{\~n}as}, Rodrigo and {Tobar}, Rodrigo J. and
         {Willis}, James S. and {Lagos}, Claudia del P. and {Power}, Chris and
         {Robotham}, Aaron S.~G.},
        title = "{Hunting for Galaxies and Halos in simulations with VELOCIraptor}",
      journal = {arXiv e-prints},
     keywords = {Astrophysics - Cosmology and Nongalactic Astrophysics},
         year = "2019",
        month = "Feb",
          eid = {arXiv:1902.01010},
        pages = {arXiv:1902.01010},
archivePrefix = {arXiv},
       eprint = {1902.01010},
 primaryClass = {astro-ph.CO},
       adsurl = {https://ui.adsabs.harvard.edu/\#abs/2019arXiv190201010E},
      adsnote = {Provided by the SAO/NASA Astrophysics Data System}
}

@techreport{Rossum1995,
 author = {Rossum, Guido},
 title = {Python Reference Manual},
 year = {1995},
 source = {http://www.ncstrl.org:8900/ncstrl/servlet/search?formname=detail\&id=oai%3Ancstrlh%3Aercim_cwi%3Aercim.cwi%2F%2FCS-R9525},
 publisher = {CWI (Centre for Mathematics and Computer Science)},
 address = {Amsterdam, The Netherlands, The Netherlands},
} 

@Misc{Scipy2001,
  author =    {Eric Jones and Travis Oliphant and Pearu Peterson and others},
  title =     {{SciPy}: Open source scientific tools for {Python}},
  year =      {2001},
  url = "http://www.scipy.org/",
  note = {[Online; accessed <today>]}
}

@Misc{Numpy2006,
  author =    {Travis Oliphant},
  title =     {{NumPy}: A guide to {NumPy}},
  year =      {2006},
  howpublished = {USA: Trelgol Publishing},
  url = "http://www.numpy.org/",
  note = {[Online; accessed <today>]}
 }

@misc{Benitez-Llambay2015,
 author       = {Alejandro Benitez-Llambay},
 title        = {py-sphviewer: Py-SPHViewer v1.0.0},
 month        = jul,
 year         = 2015,
 doi          = {10.5281/zenodo.21703},
 url          = {http://dx.doi.org/10.5281/zenodo.21703}
 }

@misc{Thompson2018,
 author       = {Robert Thompson},
 title        = {Caesar},
 month        = aug,
 year         = 2018,
 url          = {"https://bitbucket.org/rthompson/caesar/overview"}
 }

@ARTICLE{Turk2011,
   author = {{Turk}, M.~J. and {Smith}, B.~D. and {Oishi}, J.~S. and {Skory}, S. and
{Skillman}, S.~W. and {Abel}, T. and {Norman}, M.~L.},
    title = "{yt: A Multi-code Analysis Toolkit for Astrophysical Simulation Data}",
  journal = {The Astrophysical Journal Supplement Series},
archivePrefix = "arXiv",
   eprint = {1011.3514},
 primaryClass = "astro-ph.IM",
 keywords = {cosmology: theory, methods: data analysis, methods: numerical},
     year = 2011,
    month = jan,
   volume = 192,
      eid = {9},
    pages = {9},
      doi = {10.1088/0067-0049/192/1/9},
   adsurl = {http://adsabs.harvard.edu/abs/2011ApJS..192....9T},
  adsnote = {Provided by the SAO/NASA Astrophysics Data System}
}

@ARTICLE{Gill2004,
   author = {{Gill}, S.~P.~D. and {Knebe}, A. and {Gibson}, B.~K.},
    title = "{The evolution of substructure - I. A new identification method}",
  journal = {\mnras},
   eprint = {astro-ph/0404258},
 keywords = {methods: N-body simulations, methods: numerical, galaxies: formation, galaxies: haloes},
     year = 2004,
    month = jun,
   volume = 351,
    pages = {399-409},
      doi = {10.1111/j.1365-2966.2004.07786.x},
   adsurl = {http://adsabs.harvard.edu/abs/2004MNRAS.351..399G},
  adsnote = {Provided by the SAO/NASA Astrophysics Data System}
}

@ARTICLE{Knollmann2009,
   author = {{Knollmann}, S.~R. and {Knebe}, A.},
    title = "{AHF: Amiga's Halo Finder}",
  journal = {\apjs},
archivePrefix = "arXiv",
   eprint = {0904.3662},
 keywords = {methods: numerical},
     year = 2009,
    month = jun,
   volume = 182,
    pages = {608-624},
      doi = {10.1088/0067-0049/182/2/608},
   adsurl = {http://adsabs.harvard.edu/abs/2009ApJS..182..608K},
  adsnote = {Provided by the SAO/NASA Astrophysics Data System}
}

@ARTICLE{Hopkins2017,
       author = {{Hopkins}, Philip F.},
        title = "{A New Public Release of the GIZMO Code}",
      journal = {arXiv e-prints},
     keywords = {Astrophysics - Instrumentation and Methods for Astrophysics, Astrophysics - Cosmology and Nongalactic Astrophysics, Astrophysics - Earth and Planetary Astrophysics, Astrophysics - Astrophysics of Galaxies, Physics - Fluid Dynamics},
         year = "2017",
        month = "Dec",
          eid = {arXiv:1712.01294},
        pages = {arXiv:1712.01294},
archivePrefix = {arXiv},
       eprint = {1712.01294},
 primaryClass = {astro-ph.IM},
       adsurl = {https://ui.adsabs.harvard.edu/\#abs/2017arXiv171201294H},
      adsnote = {Provided by the SAO/NASA Astrophysics Data System}
}


@ARTICLE{Knabenhans2019,
   author = {{Knabenhans}, M. and {Stadel}, J. and {Marelli}, S. and {Potter}, D. and 
	{Teyssier}, R. and {Legrand}, L. and {Schneider}, A. and {Sudret}, B. and 
	{Blot}, L. and {Awan}, S. and {Burigana}, C. and {Carvalho}, C.~S. and 
	{Kurki-Suonio}, H. and {Sirri}, G.},
    title = "{Euclid preparation: II. The EUCLIDEMULATOR - a tool to compute the cosmology dependence of the nonlinear matter power spectrum}",
  journal = {\mnras},
archivePrefix = "arXiv",
   eprint = {1809.04695},
 keywords = {methods: numerical, methods: statistical, cosmological parameters, large-scale structure of Universe},
     year = 2019,
    month = apr,
   volume = 484,
    pages = {5509-5529},
      doi = {10.1093/mnras/stz197},
   adsurl = {http://adsabs.harvard.edu/abs/2019MNRAS.484.5509K},
  adsnote = {Provided by the SAO/NASA Astrophysics Data System}
}

@ARTICLE{Somerville1998,
       author = {{Somerville}, Rachel S. and {Primack}, Joel R.},
        title = "{The Star Formation History in a Hierarchical Universe}",
      journal = {arXiv e-prints},
     keywords = {Astrophysics},
         year = "1998",
        month = "Oct",
          eid = {astro-ph/9811001},
        pages = {astro-ph/9811001},
archivePrefix = {arXiv},
       eprint = {astro-ph/9811001},
 primaryClass = {astro-ph},
       adsurl = {https://ui.adsabs.harvard.edu/\#abs/1998astro.ph.11001S},
      adsnote = {Provided by the SAO/NASA Astrophysics Data System}
}

@ARTICLE{Somerville2015,
       author = {{Somerville}, Rachel S. and {Dav{\'e}}, Romeel},
        title = "{Physical Models of Galaxy Formation in a Cosmological Framework}",
      journal = {Annual Review of Astronomy and Astrophysics},
     keywords = {Astrophysics - Astrophysics of Galaxies},
         year = "2015",
        month = "Aug",
       volume = {53},
        pages = {51-113},
          doi = {10.1146/annurev-astro-082812-140951},
archivePrefix = {arXiv},
       eprint = {1412.2712},
 primaryClass = {astro-ph.GA},
       adsurl = {https://ui.adsabs.harvard.edu/\#abs/2015ARA&A..53...51S},
      adsnote = {Provided by the SAO/NASA Astrophysics Data System}
}

@ARTICLE{Naab2017,
       author = {{Naab}, Thorsten and {Ostriker}, Jeremiah P.},
        title = "{Theoretical Challenges in Galaxy Formation}",
      journal = {Annual Review of Astronomy and Astrophysics},
     keywords = {Astrophysics - Astrophysics of Galaxies},
         year = "2017",
        month = "Aug",
       volume = {55},
        pages = {59-109},
          doi = {10.1146/annurev-astro-081913-040019},
archivePrefix = {arXiv},
       eprint = {1612.06891},
 primaryClass = {astro-ph.GA},
       adsurl = {https://ui.adsabs.harvard.edu/\#abs/2017ARA&A..55...59N},
      adsnote = {Provided by the SAO/NASA Astrophysics Data System}
}

@ARTICLE{Dave2001,
       author = {{Dav{\'e}}, Romeel and {Cen}, Renyue and {Ostriker}, Jeremiah P. and
         {Bryan}, Greg L. and {Hernquist}, Lars and {Katz}, Neal and
         {Weinberg}, David H. and {Norman}, Michael L. and {O'Shea}, Brian},
        title = "{Baryons in the Warm-Hot Intergalactic Medium}",
      journal = {\apj},
     keywords = {Cosmology: Observations, Galaxies: Intergalactic Medium, Cosmology: Large-Scale Structure of Universe, Methods: Numerical, Astrophysics},
         year = "2001",
        month = "May",
       volume = {552},
        pages = {473-483},
          doi = {10.1086/320548},
archivePrefix = {arXiv},
       eprint = {astro-ph/0007217},
 primaryClass = {astro-ph},
       adsurl = {https://ui.adsabs.harvard.edu/\#abs/2001ApJ...552..473D},
      adsnote = {Provided by the SAO/NASA Astrophysics Data System}
}

@ARTICLE{Balogh2001,
       author = {{Balogh}, Michael L. and {Pearce}, Frazer R. and {Bower}, Richard G. and
         {Kay}, Scott T.},
        title = "{Revisiting the cosmic cooling crisis}",
      journal = {\mnras},
     keywords = {methods: numerical, cooling flows, galaxies: formation, Astrophysics},
         year = "2001",
        month = "Oct",
       volume = {326},
        pages = {1228-1234},
          doi = {10.1111/j.1365-2966.2001.04667.x},
archivePrefix = {arXiv},
       eprint = {astro-ph/0104041},
 primaryClass = {astro-ph},
       adsurl = {https://ui.adsabs.harvard.edu/\#abs/2001MNRAS.326.1228B},
      adsnote = {Provided by the SAO/NASA Astrophysics Data System}
}

@INPROCEEDINGS{Sturm2001,
       author = {{Sturm}, E.},
        title = "{The Nature of Ultraluminous Infrared Galaxies}",
    booktitle = {The Extragalactic Infrared Background and its Cosmological Implications},
         year = "2001",
       editor = {{Harwit}, Martin and {Hauser}, Michael G.},
       series = {IAU Symposium},
       volume = {204},
        month = "Jan",
        pages = {179},
       adsurl = {https://ui.adsabs.harvard.edu/\#abs/2001IAUS..204..179S},
      adsnote = {Provided by the SAO/NASA Astrophysics Data System}
}

@ARTICLE{Aguirre2001,
       author = {{Aguirre}, Anthony and {Hernquist}, Lars and {Schaye}, Joop and
         {Katz}, Neal and {Weinberg}, David H. and {Gardner}, Jeffrey},
        title = "{Metal Enrichment of the Intergalactic Medium in Cosmological Simulations}",
      journal = {\apj},
     keywords = {Cosmology: Theory, Galaxies: Abundances, Galaxies: Starburst, Galaxies: Intergalactic Medium, Astrophysics},
         year = "2001",
        month = "Nov",
       volume = {561},
        pages = {521-549},
          doi = {10.1086/323370},
archivePrefix = {arXiv},
       eprint = {astro-ph/0105065},
 primaryClass = {astro-ph},
       adsurl = {https://ui.adsabs.harvard.edu/\#abs/2001ApJ...561..521A},
      adsnote = {Provided by the SAO/NASA Astrophysics Data System}
}

@ARTICLE{Springel2003,
       author = {{Springel}, Volker and {Hernquist}, Lars},
        title = "{Cosmological smoothed particle hydrodynamics simulations: a hybrid multiphase model for star formation}",
      journal = {\mnras},
     keywords = {methods: numerical, galaxies: evolution, galaxies: formation, Astrophysics},
         year = "2003",
        month = "Feb",
       volume = {339},
        pages = {289-311},
          doi = {10.1046/j.1365-8711.2003.06206.x},
archivePrefix = {arXiv},
       eprint = {astro-ph/0206393},
 primaryClass = {astro-ph},
       adsurl = {https://ui.adsabs.harvard.edu/\#abs/2003MNRAS.339..289S},
      adsnote = {Provided by the SAO/NASA Astrophysics Data System}
}

@ARTICLE{Oppenheimer2006,
       author = {{Oppenheimer}, Benjamin D. and {Dav{\'e}}, Romeel},
        title = "{Cosmological simulations of intergalactic medium enrichment from galactic outflows}",
      journal = {\mnras},
     keywords = {methods: numerical, galaxies: formation, galaxies: high-redshift, intergalactic medium, cosmology: theory, Astrophysics},
         year = "2006",
        month = "Dec",
       volume = {373},
        pages = {1265-1292},
          doi = {10.1111/j.1365-2966.2006.10989.x},
archivePrefix = {arXiv},
       eprint = {astro-ph/0605651},
 primaryClass = {astro-ph},
       adsurl = {https://ui.adsabs.harvard.edu/\#abs/2006MNRAS.373.1265O},
      adsnote = {Provided by the SAO/NASA Astrophysics Data System}
}

@ARTICLE{Oppenheimer2010,
       author = {{Oppenheimer}, Benjamin D. and {Dav{\'e}}, Romeel and
         {Kere{\v{s}}}, Du{\v{s}}an and {Fardal}, Mark and {Katz}, Neal and
         {Kollmeier}, Juna A. and {Weinberg}, David H.},
        title = "{Feedback and recycled wind accretion: assembling the z = 0 galaxy mass function}",
      journal = {\mnras},
     keywords = {hydrodynamics, methods: numerical, galaxies: evolution, galaxies: formation, intergalactic medium, galaxies: luminosity function, mass function, Astrophysics - Cosmology and Nongalactic Astrophysics, Astrophysics - Astrophysics of Galaxies},
         year = "2010",
        month = "Aug",
       volume = {406},
        pages = {2325-2338},
          doi = {10.1111/j.1365-2966.2010.16872.x},
archivePrefix = {arXiv},
       eprint = {0912.0519},
 primaryClass = {astro-ph.CO},
       adsurl = {https://ui.adsabs.harvard.edu/\#abs/2010MNRAS.406.2325O},
      adsnote = {Provided by the SAO/NASA Astrophysics Data System}
}

@ARTICLE{Kauffmann1996,
       author = {{Kauffmann}, Guinevere},
        title = "{The age of elliptical galaxies and bulges in a merger model}",
      journal = {\mnras},
     keywords = {GALAXIES: ELLIPTICAL AND LENTICULAR, CD, GALAXIES: FORMATION, GALAXIES: FUNDAMENTAL PARAMETERS, GALAXIES: STELLAR CONTENT, Astrophysics},
         year = "1996",
        month = "Jul",
       volume = {281},
        pages = {487-492},
          doi = {10.1093/mnras/281.2.487},
archivePrefix = {arXiv},
       eprint = {astro-ph/9502096},
 primaryClass = {astro-ph},
       adsurl = {https://ui.adsabs.harvard.edu/\#abs/1996MNRAS.281..487K},
      adsnote = {Provided by the SAO/NASA Astrophysics Data System}
}

@ARTICLE{FrenkWhite1990,
       author = {{Frenk}, Carlos S. and {White}, Simon D.~M. and {Efstathiou}, George and
         {Davis}, Marc},
        title = "{Galaxy Clusters and the Amplitude of Primordial Fluctuations}",
      journal = {\apj},
     keywords = {Computational Astrophysics, Cosmology, Dark Matter, Galactic Clusters, Galactic Evolution, Line Of Sight, Many Body Problem, Mass To Light Ratios, Velocity Distribution, X Ray Spectra, Astrophysics, COSMOLOGY, DARK MATTER, GALAXIES: CLUSTERING, GALAXIES: REDSHIFTS, NUMERICAL METHODS},
         year = "1990",
        month = "Mar",
       volume = {351},
        pages = {10},
          doi = {10.1086/168439},
       adsurl = {https://ui.adsabs.harvard.edu/\#abs/1990ApJ...351...10F},
      adsnote = {Provided by the SAO/NASA Astrophysics Data System}
}

@ARTICLE{Maiolino2012,
       author = {{Maiolino}, R. and {Gallerani}, S. and {Neri}, R. and {Cicone}, C. and
         {Ferrara}, A. and {Genzel}, R. and {Lutz}, D. and {Sturm}, E. and
         {Tacconi}, L.~J. and {Walter}, F. and {Feruglio}, C. and {Fiore}, F. and
         {Piconcelli}, E.},
        title = "{Evidence of strong quasar feedback in the early Universe}",
      journal = {\mnras},
     keywords = {galaxies: evolution, galaxies: high-redshift, quasars: general, Astrophysics - Cosmology and Nongalactic Astrophysics, Astrophysics - Astrophysics of Galaxies},
         year = "2012",
        month = "Sep",
       volume = {425},
        pages = {L66-L70},
          doi = {10.1111/j.1745-3933.2012.01303.x},
archivePrefix = {arXiv},
       eprint = {1204.2904},
 primaryClass = {astro-ph.CO},
       adsurl = {https://ui.adsabs.harvard.edu/\#abs/2012MNRAS.425L..66M},
      adsnote = {Provided by the SAO/NASA Astrophysics Data System}
}

@ARTICLE{Navarro1995,
       author = {{Navarro}, Julio F. and {Frenk}, Carlos S. and {White}, Simon D.~M.},
        title = "{Simulations of X-ray clusters}",
      journal = {\mnras},
     keywords = {GALAXIES: CLUSTERS: GENERAL, COSMOLOGY: THEORY, DARK MATTER, X-RAYS: GENERAL, Astrophysics},
         year = "1995",
        month = "Aug",
       volume = {275},
       number = {3},
        pages = {720-740},
          doi = {10.1093/mnras/275.3.720},
archivePrefix = {arXiv},
       eprint = {astro-ph/9408069},
 primaryClass = {astro-ph},
       adsurl = {https://ui.adsabs.harvard.edu/abs/1995MNRAS.275..720N},
      adsnote = {Provided by the SAO/NASA Astrophysics Data System}
}

@ARTICLE{Lacey1994,
       author = {{Lacey}, C. and {Cole}, S.},
        title = "{Merger Rates in Hierarchical Models of Galaxy Formation - Part Two - Comparison with N-Body Simulations}",
      journal = {\mnras},
     keywords = {Astrophysics},
         year = "1994",
        month = "Dec",
       volume = {271},
        pages = {676},
          doi = {10.1093/mnras/271.3.676},
archivePrefix = {arXiv},
       eprint = {astro-ph/9402069},
 primaryClass = {astro-ph},
       adsurl = {https://ui.adsabs.harvard.edu/abs/1994MNRAS.271..676L},
      adsnote = {Provided by the SAO/NASA Astrophysics Data System}
}

@ARTICLE{More2015,
       author = {{More}, Surhud and {Diemer}, Benedikt and {Kravtsov}, Andrey V.},
        title = "{The Splashback Radius as a Physical Halo Boundary and the Growth of Halo Mass}",
      journal = {\apj},
     keywords = {cosmology: theory, dark matter, methods: numerical, Astrophysics - Cosmology and Nongalactic Astrophysics},
         year = "2015",
        month = "Sep",
       volume = {810},
       number = {1},
          eid = {36},
        pages = {36},
          doi = {10.1088/0004-637X/810/1/36},
archivePrefix = {arXiv},
       eprint = {1504.05591},
 primaryClass = {astro-ph.CO},
       adsurl = {https://ui.adsabs.harvard.edu/abs/2015ApJ...810...36M},
      adsnote = {Provided by the SAO/NASA Astrophysics Data System}
}

@article{Dehnen2012,
	title = {Improving convergence in smoothed particle hydrodynamics simulations without pairing instability},
	volume = {425},
	issn = {00358711},
	url = {http://arxiv.org/abs/1204.2471},
	doi = {10.1111/j.1365-2966.2012.21439.x},
	abstract = {The numerical convergence of smoothed particle hydrodynamics (SPH) can be severely restricted by random force errors induced by particle disorder, especially in shear flows, which are ubiquitous in astrophysics. The increase in the number NH of neighbours when switching to more extended smoothing kernels at fixed resolution (using an appropriate definition for the SPH resolution scale) is insufficient to combat these errors. Consequently, trading resolution for better convergence is necessary, but for traditional smoothing kernels this option is limited by the pairing (or clumping) instability. Therefore, we investigate the suitability of the Wendland functions as smoothing kernels and compare them with the traditional B-splines. Linear stability analysis in three dimensions and test simulations demonstrate that the Wendland kernels avoid the pairing instability for all NH, despite having vanishing derivative at the origin (disproving traditional ideas about the origin of this instability; instead, we uncover a relation with the kernel Fourier transform and give an explanation in terms of the SPH density estimator). The Wendland kernels are computationally more convenient than the higher-order B-splines, allowing large NH and hence better numerical convergence (note that computational costs rise sub-linear with NH). Our analysis also shows that at low NH the quartic spline kernel with NH {\textasciitilde}= 60 obtains much better convergence then the standard cubic spline.},
	number = {2},
	urldate = {2019-06-04},
	journal = {Monthly Notices of the Royal Astronomical Society},
	author = {Dehnen, Walter and Aly, Hossam},
	month = sep,
	year = {2012},
	note = {arXiv: 1204.2471},
	keywords = {Astrophysics - Instrumentation and Methods for Astrophysics, Physics - Fluid Dynamics, Physics - Computational Physics},
	pages = {1068--1082},
	annote = {Comment: substantially revised version, accepted for publication in MNRAS, 15 pages, 13 figures},
	file = {arXiv\:1204.2471 PDF:/Users/mphf18/Zotero/storage/CUE5H3U4/Dehnen and Aly - 2012 - Improving convergence in smoothed particle hydrody.pdf:application/pdf;arXiv.org Snapshot:/Users/mphf18/Zotero/storage/68CSBWI5/1204.html:text/html}
}

@article{Bryan1998,
	title = {Statistical {Properties} of {X}‐{Ray} {Clusters}: {Analytic} and {Numerical} {Comparisons}},
	volume = {495},
	issn = {0004-637X, 1538-4357},
	shorttitle = {Statistical {Properties} of {X}‐{Ray} {Clusters}},
	url = {http://stacks.iop.org/0004-637X/495/i=1/a=80},
	doi = {10.1086/305262},
	abstract = {We compare the results of Eulerian hydrodynamic simulations of cluster formation against virial scaling relations between four bulk quantities : the cluster mass, the dark matter velocity dispersion, the gas temperature, and the cluster luminosity. The comparison is made for a large number of clusters at a range of redshifts in three di†erent cosmological models (cold plus hot dark matter, cold dark matter, and open cold dark matter). We Ðnd that the analytic formulae provide a good description of the relations between three of the four numerical quantities. The fourth (luminosity) also agrees once we introduce a procedure to correct for the Ðxed numerical resolution. We also compute the normalizations for the virial relations and compare extensively to the existing literature, Ðnding remarkably good agreement.},
	language = {en},
	number = {1},
	urldate = {2019-07-08},
	journal = {\apj},
	author = {Bryan, Greg L. and Norman, Michael L.},
	month = mar,
	year = {1998},
	pages = {80--99},
	file = {Bryan and Norman - 1998 - Statistical Properties of X‐Ray Clusters Analytic.pdf:/Users/mphf18/Zotero/storage/FKTVW85D/Bryan and Norman - 1998 - Statistical Properties of X‐Ray Clusters Analytic.pdf:application/pdf}
}


@article{Mansfield2017,
	title = {Splashback {Shells} of {Cold} {Dark} {Matter} {Halos}},
	volume = {841},
	issn = {1538-4357},
	url = {http://stacks.iop.org/0004-637X/841/i=1/a=34?key=crossref.d7ae313c947fc5ebcdae8a3782ad31de},
	doi = {10.3847/1538-4357/aa7047},
	abstract = {The density field in the outskirts of dark matter halos is discontinuous as a result of a caustic formed by matter at its first apocenter after infall. In this paper, we present an algorithm to identify the “splashback shell” formed by these apocenters in individual simulated halos using only a single snapshot of the density field. We implement this algorithm in the code SHELLFISH (SHELL Finding In Spheroidal Halos) and demonstrate that the code identifies splashback shells correctly and measures their properties with an accuracy of {\textless}5\% for halos with more than 50,000 particles and mass accretion rates of GDK14 {\textgreater} 0.5. Using SHELLFISH, we present the first estimates for several basic pdwriisotthpriebirnutciterioseanosiofnfginthGdeDisvKei1d4qu,uabaluntstipwtileaessshahbsoafwcuknthcstahitoe, lnilnss,doesfpuGecDnhKd1ea4ns, tnro2a0fd0iamuc,sc,arneRtdisopz,n.},
	language = {en},
	number = {1},
	urldate = {2019-07-09},
	journal = {The Astrophysical Journal},
	author = {Mansfield, Philip and Kravtsov, Andrey V. and Diemer, Benedikt},
	month = may,
	year = {2017},
	pages = {34},
	file = {Mansfield et al. - 2017 - Splashback Shells of Cold Dark Matter Halos.pdf:/Users/mphf18/Zotero/storage/Z55V7BUG/Mansfield et al. - 2017 - Splashback Shells of Cold Dark Matter Halos.pdf:application/pdf}
}


@article{Diemer2017,
	title = {The {Splashback} {Radius} of {Halos} from {Particle} {Dynamics}. {II}. {Dependence} on {Mass}, {Accretion} {Rate}, {Redshift}, and {Cosmology}},
	volume = {843},
	issn = {1538-4357},
	url = {http://stacks.iop.org/0004-637X/843/i=2/a=140?key=crossref.89fbaecd9a7d79852960610c249d5793},
	doi = {10.3847/1538-4357/aa79ab},
	abstract = {The splashback radius Rsp, the apocentric radius of particles on their first orbit after falling into a dark matter halo, has recently been suggested to be a physically motivated halo boundary that separates accreting from orbiting material. Using the SPARTA code presented in Paper I, we analyze the orbits of billions of particles in cosmological simulations of structure formation and measure Rsp for a large sample of halos that span a mass range from dwarf galaxy to massive cluster halos, reach redshift 8, and include WMAP, Planck, and self-similar cosmologies. We analyze the dependence of Rsp/R200m and Msp/M200m on the mass accretion rate Γ, halo mass, redshift, and cosmology. The scatter in these relations varies between 0.02 and 0.1 dex. While we confirm the known trend that Rsp/R200m decreases with Γ, the relationships turn out to be more complex than previously thought, demonstrating that Rsp is an independent definition of the halo boundary that cannot trivially be reconstructed from spherical overdensity definitions. We present fitting functions for Rsp/R200m and Msp/M200m as a function of accretion rate, peak height, and redshift, achieving an accuracy of 5\% or better everywhere in the parameter space explored. We discuss the physical meaning of the distribution of particle apocenters and show that the previously proposed definition of Rsp as the radius of the steepest logarithmic density slope encloses roughly three-quarters of the apocenters. Finally, we conclude that no analytical model presented thus far can fully explain our results.},
	language = {en},
	number = {2},
	urldate = {2019-07-09},
	journal = {The Astrophysical Journal},
	author = {Diemer, Benedikt and Mansfield, Philip and Kravtsov, Andrey V. and More, Surhud},
	month = jul,
	year = {2017},
	pages = {140},
	file = {Diemer et al. - 2017 - The Splashback Radius of Halos from Particle Dynam.pdf:/Users/mphf18/Zotero/storage/JC42YQ45/Diemer et al. - 2017 - The Splashback Radius of Halos from Particle Dynam.pdf:application/pdf}
}

@article{Diemer2017a,
	title = {The {Splashback} {Radius} of {Halos} from {Particle} {Dynamics}. {I}. {The} {SPARTA} {Algorithm}},
	volume = {231},
	issn = {1538-4365},
	url = {http://stacks.iop.org/0067-0049/231/i=1/a=5?key=crossref.834d28473a197d71497b48809820e2eb},
	doi = {10.3847/1538-4365/aa799c},
	abstract = {Motivated by the recent proposal of the splashback radius as a physical boundary of dark-matter halos, we present a parallel computer code for Subhalo and PARticle Trajectory Analysis (SPARTA). The code analyzes the orbits of all simulation particles in all host halos, billions of orbits in the case of typical cosmological N-body simulations. Within this general framework, we develop an algorithm that accurately extracts the location of the first apocenter of particles after infall into a halo, or splashback. We define the splashback radius of a halo as the smoothed average of the apocenter radii of individual particles. This definition allows us to reliably measure the splashback radii of 95\% of host halos above a resolution limit of 1000 particles. We show that, on average, the splashback radius and mass are converged to better than 5\% accuracy with respect to mass resolution, snapshot spacing, and all free parameters of the method.},
	language = {en},
	number = {1},
	urldate = {2019-07-09},
	journal = {The Astrophysical Journal Supplement Series},
	author = {Diemer, Benedikt},
	month = jul,
	year = {2017},
	pages = {5},
	file = {Diemer - 2017 - The Splashback Radius of Halos from Particle Dynam.pdf:/Users/mphf18/Zotero/storage/MMEAIG3B/Diemer - 2017 - The Splashback Radius of Halos from Particle Dynam.pdf:application/pdf}
}


@article{Onorbe2014,
	title = {How to zoom: bias, contamination and {Lagrange} volumes in multimass cosmological simulations},
	volume = {437},
	issn = {0035-8711, 1365-2966},
	shorttitle = {How to zoom},
	url = {https://academic.oup.com/mnras/article-lookup/doi/10.1093/mnras/stt2020},
	doi = {10.1093/mnras/stt2020},
	abstract = {We perform a suite of multimass cosmological zoom simulations of individual dark matter haloes and explore how to best select Lagrangian regions for resimulation without contaminating the halo of interest with low-resolution particles. Such contamination can lead to significant errors in the gas distribution of hydrodynamical simulations, as we show. For a fixed Lagrange volume, we find that the chance of contamination increases systematically with the level of zoom. In order to avoid contamination, the Lagrangian volume selected for resimulation must increase monotonically with the resolution difference between parent box and the zoom region. We provide a simple formula for selecting Lagrangian regions (in units of the halo virial volume) as a function of the level of zoom required. We also explore the degree to which a halo’s Lagrangian volume correlates with other halo properties (concentration, spin, formation time, shape, etc.) and find no significant correlation. There is a mild correlation between Lagrange volume and environment, such that haloes living in the most clustered regions have larger Lagrangian volumes. Nevertheless, selecting haloes to be isolated is not the best way to ensure inexpensive zoom simulations. We explain how one can safely choose haloes with the smallest Lagrangian volumes, which are the least expensive to resimulate, without biasing one’s sample.},
	language = {en},
	number = {2},
	urldate = {2019-07-09},
	journal = {Monthly Notices of the Royal Astronomical Society},
	author = {Onorbe, J. and Garrison-Kimmel, S. and Maller, A. H. and Bullock, J. S. and Rocha, M. and Hahn, O.},
	month = jan,
	year = {2014},
	pages = {1894--1908},
	file = {Onorbe et al. - 2014 - How to zoom bias, contamination and Lagrange volu.pdf:/Users/mphf18/Zotero/storage/4KUUI7M3/Onorbe et al. - 2014 - How to zoom bias, contamination and Lagrange volu.pdf:application/pdf}
}


@article{Angles-Alcazar2014,
	title = {{COSMOLOGICAL} {ZOOM} {SIMULATIONS} {OF} \textit{z} = 2 {GALAXIES}: {THE} {IMPACT} {OF} {GALACTIC} {OUTFLOWS}},
	volume = {782},
	issn = {0004-637X, 1538-4357},
	shorttitle = {{COSMOLOGICAL} {ZOOM} {SIMULATIONS} {OF} \textit{z} = 2 {GALAXIES}},
	url = {http://stacks.iop.org/0004-637X/782/i=2/a=84?key=crossref.e2e50d9da5cf0ba4f27a00b913a0f545},
	doi = {10.1088/0004-637X/782/2/84},
	abstract = {We use high-resolution cosmological zoom simulations with ∼200 pc resolution at z = 2 and various prescriptions for galactic outflows in order to explore the impact of winds on the morphological, dynamical, and structural properties of eight individual galaxies with halo masses ∼1011–2 × 1012 M at z = 2. We present a detailed comparison to spatially and spectrally resolved H α and other observations of z ≈ 2 galaxies. We find that simulations without winds produce massive, compact galaxies with low gas fractions, super-solar metallicities, high bulge fractions, and much of the star formation concentrated within the inner kiloparsec. Strong winds are required to maintain high gas fractions, redistribute star-forming gas over larger scales, and increase the velocity dispersion of simulated galaxies, more in agreement with the large, extended, turbulent disks typical of high-redshift star-forming galaxies. Winds also suppress early star formation to produce high-redshift cosmic star formation efficiencies in better agreement with observations. Sizes, rotation velocities, and velocity dispersions all scale with stellar mass in accord with observations. Our simulations produce a diversity of morphological characteristics—among our three most massive galaxies, we find a quiescent grand-design spiral, a very compact star-forming galaxy, and a clumpy disk undergoing a minor merger; the clumps are evident in Hα but not in the stars. Rotation curves are generally slowly rising, particularly when calculated using azimuthal velocities rather than enclosed mass. Our results are broadly resolution-converged. These results show that cosmological simulations including outflows can produce disk galaxies similar to those observed during the peak epoch of cosmic galaxy growth.},
	language = {en},
	number = {2},
	urldate = {2019-07-09},
	journal = {The Astrophysical Journal},
	author = {Anglés-Alcázar, Daniel and Davé, Romeel and Özel, Feryal and Oppenheimer, Benjamin D.},
	month = jan,
	year = {2014},
	pages = {84},
	file = {Anglés-Alcázar et al. - 2014 - COSMOLOGICAL ZOOM SIMULATIONS OF izi = 2 GALA.pdf:/Users/mphf18/Zotero/storage/6FVMW9BG/Anglés-Alcázar et al. - 2014 - COSMOLOGICAL ZOOM SIMULATIONS OF izi = 2 GALA.pdf:application/pdf}
}


@article{Hellwing2016,
	title = {The effect of baryons on redshift space distortions and cosmic density and velocity fields in the {EAGLE} simulation},
	volume = {461},
	issn = {1745-3925, 1745-3933},
	url = {https://academic.oup.com/mnrasl/article-lookup/doi/10.1093/mnrasl/slw081},
	doi = {10.1093/mnrasl/slw081},
	abstract = {We use the Evolution and Assembly of GaLaxies and their Environments (EAGLE) galaxy formation simulation to study the effects of baryons on the power spectrum of the total matter and dark matter distributions and on the velocity fields of dark matter and galaxies. On scales k 4 h Mpc−1 the effect of baryons on the amplitude of the total matter power spectrum is greater than 1 per cent. The back-reaction of baryons affects the density field of the dark matter at the level of ∼3 per cent on scales of 1 ≤ k/( h Mpc−1) ≤ 5. The dark matter velocity divergence power spectrum at k 0.5 h Mpc−1 is changed by less than 1 per cent. The 2D redshift space power spectrum is affected at the level of ∼6 per cent at {\textbar}k{\textbar} 1 h Mpc−1 (for μ {\textgreater} 0.5), but for {\textbar}k{\textbar} ≤ 0.4 h Mpc−1 it differs by less than 1 per cent. We report vanishingly small baryonic velocity bias for haloes: the peculiar velocities of haloes with M200 {\textgreater} 3 × 1011 M (hosting galaxies with M∗ {\textgreater} 109 M ) are affected at the level of at most 1 km s−1, which is negligible for 1 per cent-precision cosmology. We caution that since EAGLE overestimates cluster gas fractions it may also underestimate the impact of baryons, particularly for the total matter power spectrum. Nevertheless, our findings suggest that for theoretical modelling of redshift space distortions and galaxy velocity-based statistics, baryons and their back-reaction can be safely ignored at the current level of observational accuracy. However, we confirm that the modelling of the total matter power spectrum in weak lensing studies needs to include realistic galaxy formation physics in order to achieve the accuracy required in the precision cosmology era.},
	language = {en},
	number = {1},
	urldate = {2019-07-09},
	journal = {Monthly Notices of the Royal Astronomical Society: Letters},
	author = {Hellwing, Wojciech A. and Schaller, Matthieu and Frenk, Carlos S. and Theuns, Tom and Schaye, Joop and Bower, Richard G. and Crain, Robert A.},
	month = sep,
	year = {2016},
	pages = {L11--L15},
	file = {Hellwing et al. - 2016 - The effect of baryons on redshift space distortion.pdf:/Users/mphf18/Zotero/storage/65KNVPDQ/Hellwing et al. - 2016 - The effect of baryons on redshift space distortion.pdf:application/pdf}
}


@article{Diemer2014,
	title = {{DEPENDENCE} {OF} {THE} {OUTER} {DENSITY} {PROFILES} {OF} {HALOS} {ON} {THEIR} {MASS} {ACCRETION} {RATE}},
	volume = {789},
	issn = {0004-637X, 1538-4357},
	url = {http://stacks.iop.org/0004-637X/789/i=1/a=1?key=crossref.ef3e8c1e4dd1740f94e8321cd77238cf},
	doi = {10.1088/0004-637X/789/1/1},
	abstract = {We present a systematic study of the density profiles of ΛCDM halos, focusing on the outer regions, 0.1 {\textless} r/Rvir {\textless} 9. We show that the median and mean profiles of halo samples of a given peak height exhibit significant deviations from the universal analytic profiles discussed previously in the literature, such as the Navarro–Frenk–White and Einasto profiles, at radii r 0.5R200m. In particular, at these radii the logarithmic slope of the median density profiles of massive or rapidly accreting halos steepens more sharply than predicted. The steepest slope of the profiles occurs at r ≈ R200m, and its absolute value increases with increasing peak height or mass accretion rate, reaching slopes of −4 and steeper. Importantly, we find that the outermost density profiles at r R200m are remarkably self-similar when radii are rescaled by R200m. This self-similarity indicates that radii defined with respect to the mean density are preferred for describing the structure and evolution of the outer profiles. However, the inner density profiles are most self-similar when radii are rescaled by R200c. We propose a new fitting formula that describes the median and mean profiles of halo samples selected by their peak height or mass accretion rate with accuracy 10\% at all radii, redshifts, and masses we studied, r 9Rvir, 0 {\textless} z {\textless} 6, and Mvir {\textgreater} 1.7 × 1010 h−1 M . We discuss observational signatures of the profile features described above and show that the steepening of the outer profile should be detectable in future weak-lensing analyses of massive clusters. Such observations could be used to estimate the mass accretion rate of cluster halos.},
	language = {en},
	number = {1},
	urldate = {2019-07-09},
	journal = {The Astrophysical Journal},
	author = {Diemer, Benedikt and Kravtsov, Andrey V.},
	month = jun,
	year = {2014},
	pages = {1},
	file = {Diemer and Kravtsov - 2014 - DEPENDENCE OF THE OUTER DENSITY PROFILES OF HALOS .pdf:/Users/mphf18/Zotero/storage/4R7U3GJI/Diemer and Kravtsov - 2014 - DEPENDENCE OF THE OUTER DENSITY PROFILES OF HALOS .pdf:application/pdf}
}

@article{Adhikari2014,
	title = {Splashback in accreting dark matter halos},
	volume = {2014},
	issn = {1475-7516},
	url = {http://stacks.iop.org/1475-7516/2014/i=11/a=019?key=crossref.ee44abd4d5e08da3b2d3a6cb72828077},
	doi = {10.1088/1475-7516/2014/11/019},
	abstract = {Recent work has shown that density profiles in the outskirts of dark matter halos can become extremely steep over a narrow range of radius. This behavior is produced by splashback material on its first apocentric passage after accretion. We show that the location of this splashback feature may be understood quite simply, from first principles. We present a simple model, based on spherical collapse, that accurately predicts the location of splashback without any free parameters. The important quantities that determine the splashback radius are accretion rate and redshift.},
	language = {en},
	number = {11},
	urldate = {2019-07-09},
	journal = {Journal of Cosmology and Astroparticle Physics},
	author = {Adhikari, Susmita and Dalal, Neal and Chamberlain, Robert T.},
	month = nov,
	year = {2014},
	pages = {019--019},
	file = {Adhikari et al. - 2014 - Splashback in accreting dark matter halos.pdf:/Users/mphf18/Zotero/storage/J55UD5YF/Adhikari et al. - 2014 - Splashback in accreting dark matter halos.pdf:application/pdf}
}


@article{Somerville2015b,
	title = {Star formation in semi-analytic galaxy formation models with multiphase gas},
	volume = {453},
	issn = {0035-8711, 1365-2966},
	url = {https://academic.oup.com/mnras/article-lookup/doi/10.1093/mnras/stv1877},
	doi = {10.1093/mnras/stv1877},
	abstract = {We implement physically motivated recipes for partitioning cold gas into different phases (atomic, molecular, and ionized) in galaxies within semi-analytic models of galaxy formation based on cosmological merger trees. We then model the conversion of molecular gas into stars using empirical recipes motivated by recent observations. We explore the impact of these new recipes on the evolution of fundamental galaxy properties such as stellar mass, star formation rate (SFR), and gas and stellar phase metallicity. We present predictions for stellar mass functions, stellar mass versus SFR relations, and cold gas phase and stellar mass–metallicity relations for our fiducial models, from redshift z ∼ 6 to the present day. In addition we present predictions for the global SFR, mass assembly history, and cosmic enrichment history. We find that the predicted stellar properties of galaxies (stellar mass, SFR, metallicity) are remarkably insensitive to the details of the recipes used for partitioning gas into H I and H2. We see significant sensitivity to the recipes for H2 formation only in very low mass haloes (Mh 1010.5 M ), which host galaxies with stellar masses m∗ 108 M . The properties of low-mass galaxies are also quite insensitive to the details of the recipe used for converting H2 into stars, while the formation epoch of massive galaxies does depend on this significantly. We argue that this behaviour can be interpreted within the framework of a simple equilibrium model for galaxy evolution, in which the conversion of cold gas into stars is balanced on average by inflows and outflows.},
	language = {en},
	number = {4},
	urldate = {2019-07-09},
	journal = {Monthly Notices of the Royal Astronomical Society},
	author = {Somerville, Rachel S. and Popping, Gergö and Trager, Scott C.},
	month = nov,
	year = {2015},
	pages = {4338--4368},
	file = {Somerville et al. - 2015 - Star formation in semi-analytic galaxy formation m.pdf:/Users/mphf18/Zotero/storage/EDAIY5FJ/Somerville et al. - 2015 - Star formation in semi-analytic galaxy formation m.pdf:application/pdf}
}


@article{Christensen2018,
	title = {Tracing {Outflowing} {Metals} in {Simulations} of {Dwarf} and {Spiral} {Galaxies}},
	volume = {867},
	issn = {1538-4357},
	url = {http://stacks.iop.org/0004-637X/867/i=2/a=142?key=crossref.a74ed0a1dbb10e30ccb06bd1c0ed73dc},
	doi = {10.3847/1538-4357/aae374},
	abstract = {We analyze the metal accumulation in dwarf and spiral galaxies by following the history of metal enrichment and outflows in a suite of 20 high-resolution simulated galaxies. These simulations agree with the observed stellar and gas-phase mass–metallicity relation, an agreement that relies on large fractions of the produced metals escaping into the circumgalactic media. For instance, in galaxies with Mvir∼109.5–1010 M, we find that about ∼85\% of the available metals are outside of the galactic disk at z=0, although the fraction decreases to a little less than half in Milky-Way-mass galaxies. In many cases, these metals are spread far beyond the virial radius. We analyze the metal deficit within the ISM and stars in the context of previous work tracking the inflow and outflow of baryons. Outflows are prevalent across the entire mass range, as is reaccretion. We find that between 40\% and 80\% of all metals removed from the galactic disk are later reaccreted. The outflows themselves are metal-enriched relative to the ISM by a factor of 0.2 dex because of the correspondence between sites of metal enrichment and outflows. As a result, the metal mass loading factor scales as hmetals µ vc-ir0c.91, a somewhat shallower scaling than the total mass loading factor. We analyze the simulated galaxies within the context of analytic chemical evolution models by determining their net metal expulsion efficiencies, which encapsulate the rates of metal loss and reaccretion. We discuss these results in light of the inflow and outflow properties necessary for reproducing the mass–metallicity relation.},
	language = {en},
	number = {2},
	urldate = {2019-07-09},
	journal = {The Astrophysical Journal},
	author = {Christensen, Charlotte R. and Davé, Romeel and Brooks, Alyson and Quinn, Thomas and Shen, Sijing},
	month = nov,
	year = {2018},
	pages = {142},
	file = {Christensen et al. - 2018 - Tracing Outflowing Metals in Simulations of Dwarf .pdf:/Users/mphf18/Zotero/storage/UIBRQ8XP/Christensen et al. - 2018 - Tracing Outflowing Metals in Simulations of Dwarf .pdf:application/pdf}
}


@article{Christensen2016,
	title = {{IN}-{N}-{OUT}: {THE} {GAS} {CYCLE} {FROM} {DWARFS} {TO} {SPIRAL} {GALAXIES}},
	volume = {824},
	issn = {1538-4357},
	shorttitle = {{IN}-{N}-{OUT}},
	url = {http://stacks.iop.org/0004-637X/824/i=1/a=57?key=crossref.405568dfaad73a8f714f8d321774300c},
	doi = {10.3847/0004-637X/824/1/57},
	abstract = {We examine the scalings of galactic outflows with halo mass across a suite of 20 high-resolution cosmological zoom galaxy simulations covering halo masses in the range 109.5–1012 M. These simulations self-consistently generate outflows from the available supernova energy in a manner that successfully reproduces key galaxy observables, including the stellar mass–halo mass, Tully–Fisher, and mass–metallicity relations. We quantify the importance of ejective feedback to setting the stellar mass relative to the efficiency of gas accretion and star formation. Ejective feedback is increasingly important as galaxy mass decreases; we find an effective mass loading factor that scales as vc-ir2c.2, with an amplitude and shape that are invariant with redshift. These scalings are consistent with analytic models for energy-driven wind, based solely on the halo potential. Recycling is common: about half of the outflow mass across all galaxy masses is later reaccreted. The recycling timescale is typically ∼1 Gyr, virtually independent of halo mass. Recycled material is reaccreted farther out in the disk and with typically ∼2–3 times more angular momentum. These results elucidate and quantify how the baryon cycle plausibly regulates star formation and alters the angular momentum distribution of disk material across the halo mass range where most cosmic star formation occurs.},
	language = {en},
	number = {1},
	urldate = {2019-07-09},
	journal = {The Astrophysical Journal},
	author = {Christensen, Charlotte R. and Davé, Romeel and Governato, Fabio and Pontzen, Andrew and Brooks, Alyson and Munshi, Ferah and Quinn, Thomas and Wadsley, James},
	month = jun,
	year = {2016},
	pages = {57},
	file = {Christensen et al. - 2016 - IN-N-OUT THE GAS CYCLE FROM DWARFS TO SPIRAL GALA.pdf:/Users/mphf18/Zotero/storage/4N6UT5TZ/Christensen et al. - 2016 - IN-N-OUT THE GAS CYCLE FROM DWARFS TO SPIRAL GALA.pdf:application/pdf}
}


@article{Hafen2017,
	title = {Low-redshift {Lyman} limit systems as diagnostics of cosmological inflows and outflows},
	volume = {469},
	issn = {0035-8711, 1365-2966},
	url = {https://academic.oup.com/mnras/article/469/2/2292/3747508},
	doi = {10.1093/mnras/stx952},
	abstract = {We use cosmological hydrodynamic simulations with stellar feedback from the FIRE (Feedback In Realistic Environments) project to study the physical nature of Lyman limit systems (LLSs) at z ≤ 1. At these low redshifts, LLSs are closely associated with dense gas structures surrounding galaxies, such as galactic winds, dwarf satellites and cool inflows from the intergalactic medium. Our analysis is based on 14 zoom-in simulations covering the halo mass range Mh ≈ 109–1013 M at z = 0, which we convolve with the dark matter halo mass function to produce cosmological statistics. We find that the majority of cosmologically selected LLSs are associated with haloes in the mass range 1010 Mh 1012 M . The incidence and H I column density distribution of simulated absorbers with columns in the range 1016.2 ≤ NH I ≤ 2 × 1020 cm−2 are consistent with observations. High-velocity outflows (with radial velocity exceeding the halo circular velocity by a factor of 2) tend to have higher metallicities ([X/H] ∼ −0.5) while very low metallicity ([X/H] {\textless} −2) LLSs are typically associated with gas infalling from the intergalactic medium. However, most LLSs occupy an intermediate region in metallicity-radial velocity space, for which there is no clear trend between metallicity and radial kinematics. The overall simulated LLS metallicity distribution has a mean (standard deviation) [X/H] = −0.9 (0.4) and does not show significant evidence for bimodality, in contrast to recent observational studies, but consistent with LLSs arising from haloes with a broad range of masses and metallicities.},
	language = {en},
	number = {2},
	urldate = {2019-07-09},
	journal = {Monthly Notices of the Royal Astronomical Society},
	author = {Hafen, Zachary and Faucher-Giguère, Claude-André and Anglés-Alcázar, Daniel and Kereš, Dušan and Feldmann, Robert and Chan, T. K. and Quataert, Eliot and Murray, Norman and Hopkins, Philip F.},
	month = aug,
	year = {2017},
	pages = {2292--2304},
	file = {Hafen et al. - 2017 - Low-redshift Lyman limit systems as diagnostics of.pdf:/Users/mphf18/Zotero/storage/W5CD5F8F/Hafen et al. - 2017 - Low-redshift Lyman limit systems as diagnostics of.pdf:application/pdf}
}


@article{VandenBosch2018,
	title = {Dark matter substructure in numerical simulations: a tale of discreteness noise, runaway instabilities, and artificial disruption},
	volume = {475},
	issn = {0035-8711, 1365-2966},
	shorttitle = {Dark matter substructure in numerical simulations},
	url = {https://academic.oup.com/mnras/article/475/3/4066/4797185},
	doi = {10.1093/mnras/sty084},
	abstract = {To gain understanding of the complicated, non-linear, and numerical processes associated with the tidal evolution of dark matter subhaloes in numerical simulation, we perform a large suite of idealized simulations that follow individual N-body subhaloes in a fixed, analytical host halo potential. By varying both physical and numerical parameters, we investigate under what conditions the subhaloes undergo disruption. We confirm the conclusions from our more analytical assessment in van den Bosch et al. that most disruption is numerical in origin; as long as a subhalo is resolved with sufficient mass and force resolution, a bound remnant survives. This implies that state-of-the-art cosmological simulations still suffer from significant overmerging. We demonstrate that this is mainly due to inadequate force softening, which causes excessive mass loss and artificial tidal disruption. In addition, we show that subhaloes in N-body simulations are susceptible to a runaway instability triggered by the amplification of discreteness noise in the presence of a tidal field. These two processes conspire to put serious limitations on the reliability of dark matter substructure in state-of-the-art cosmological simulations. We present two criteria that can be used to assess whether individual subhaloes in cosmological simulations are reliable or not, and advocate that subhaloes that satisfy either of these two criteria be discarded from further analysis. We discuss the potential implications of this work for several areas in astrophysics.},
	language = {en},
	number = {3},
	urldate = {2019-07-09},
	journal = {Monthly Notices of the Royal Astronomical Society},
	author = {van den Bosch, Frank C and Ogiya, Go},
	month = apr,
	year = {2018},
	pages = {4066--4087},
	file = {van den Bosch and Ogiya - 2018 - Dark matter substructure in numerical simulations.pdf:/Users/mphf18/Zotero/storage/WNFDTZ8L/van den Bosch and Ogiya - 2018 - Dark matter substructure in numerical simulations.pdf:application/pdf}
}


@article{Greene2012,
	title = {A {SPECTACULAR} {OUTFLOW} {IN} {AN} {OBSCURED} {QUASAR}},
	volume = {746},
	issn = {0004-637X, 1538-4357},
	url = {http://stacks.iop.org/0004-637X/746/i=1/a=86?key=crossref.c507e1b74f8004d5bab3de34df83bb1e},
	doi = {10.1088/0004-637X/746/1/86},
	abstract = {SDSS J1356 + 1026 is a pair of interacting galaxies at redshift z = 0.123 that hosts a luminous obscured quasar in its northern nucleus. Here we present two long-slit Magellan LDSS-3 spectra that reveal a pair of symmetric ∼10 kpc size outflows emerging from this nucleus, with observed expansion velocities of ∼250 km s−1 in projection. We present a kinematic model of these outflows and argue that the deprojected physical velocities of expansion are likely ∼1000 km s−1 and that the kinetic energy of the expanding shells is likely 1044–45 erg s−1, with an absolute minimum of {\textgreater}1042 erg s−1. Although a radio counterpart is detected at 1.4 GHz, it is faint enough that the quasar is considered to be radio quiet by all standard criteria, and there is no evidence of extended emission due to radio lobes, whether aged or continuously powered by an ongoing jet. We argue that the likely level of star formation is insufficient to power the observed energetic outflow and that SDSS J1356 + 1026 is a good case for radio-quiet quasar feedback. In further support of this hypothesis, polarimetric observations show that the direction of quasar illumination is coincident with the direction of the outflow.},
	language = {en},
	number = {1},
	urldate = {2019-07-09},
	journal = {The Astrophysical Journal},
	author = {Greene, Jenny E. and Zakamska, Nadia L. and Smith, Paul S.},
	month = feb,
	year = {2012},
	pages = {86},
	file = {Greene et al. - 2012 - A SPECTACULAR OUTFLOW IN AN OBSCURED QUASAR.pdf:/Users/mphf18/Zotero/storage/9RUUWMFD/Greene et al. - 2012 - A SPECTACULAR OUTFLOW IN AN OBSCURED QUASAR.pdf:application/pdf}
}

@article{Zakamska2016,
	title = {Discovery of extreme [{O} iii] λ5007 Å outflows in high-redshift red quasars},
	volume = {459},
	issn = {0035-8711, 1365-2966},
	url = {https://academic.oup.com/mnras/article-lookup/doi/10.1093/mnras/stw718},
	doi = {10.1093/mnras/stw718},
	abstract = {Black hole feedback is now a standard component of galaxy formation models. These models predict that the impact of black hole activity on its host galaxy likely peaked at z = 2–3, the epoch of strongest star formation activity and black hole accretion activity in the Universe. We used XSHOOTER on the Very Large Telescope to measure rest-frame optical spectra of four z ∼ 2.5 extremely red quasars with infrared luminosities ∼1047 erg s−1. We present the discovery of very broad (full width at half max = 2600–5000 km s−1), strongly blueshifted (by up to 1500 km s−1) [O III] λ5007 Å emission lines in these objects. In a large sample of type 2 and red quasars, [O III] kinematics are positively correlated with infrared luminosity, and the four objects in our sample are on the extreme end in both [O III] kinematics and infrared luminosity. We estimate that at least 3 per cent of the bolometric luminosity in these objects is being converted into the kinetic power of the observed wind. Photo-ionization estimates suggest that the [O III] emission might be extended on a few kpc scales, which would suggest that the extreme outflow is affecting the entire host galaxy of the quasar. These sources may be the signposts of the most extreme form of quasar feedback at the peak epoch of galaxy formation, and may represent an active ‘blow-out’ phase of quasar evolution.},
	language = {en},
	number = {3},
	urldate = {2019-07-09},
	journal = {Monthly Notices of the Royal Astronomical Society},
	author = {Zakamska, Nadia L. and Hamann, Fred and Pâris, Isabelle and Brandt, W. N. and Greene, Jenny E. and Strauss, Michael A. and Villforth, Carolin and Wylezalek, Dominika and Alexandroff, Rachael M. and Ross, Nicholas P.},
	month = jul,
	year = {2016},
	pages = {3144--3160},
	file = {Zakamska et al. - 2016 - Discovery of extreme [O iii] λ5007 Å outflows in h.pdf:/Users/mphf18/Zotero/storage/J9VCCK8Y/Zakamska et al. - 2016 - Discovery of extreme [O iii] λ5007 Å outflows in h.pdf:application/pdf}
}


@article{Fabian2012,
	title = {Observational {Evidence} of {Active} {Galactic} {Nuclei} {Feedback}},
	volume = {50},
	issn = {0066-4146, 1545-4282},
	url = {http://www.annualreviews.org/doi/10.1146/annurev-astro-081811-125521},
	doi = {10.1146/annurev-astro-081811-125521},
	abstract = {Radiation, winds, and jets from the active nucleus of a massive galaxy can interact with its interstellar medium, and this can lead to ejection or heating of the gas. This terminates star formation in the galaxy and stifles accretion onto the black hole. Such active galactic nuclei (AGN) feedback can account for the observed proportionality between the central black hole and the host galaxy mass. Direct observational evidence for the radiative or quasar mode of feedback, which occurs when AGN are very luminous, has been difficult to obtain but is accumulating from a few exceptional objects. Feedback from the kinetic or radio mode, which uses the mechanical energy of radioemitting jets often seen when AGN are operating at a lower level, is common in massive elliptical galaxies. This mode is well observed directly through X-ray observations of the central galaxies of cool core clusters in the form of bubbles in the hot surrounding medium. The energy flow, which is roughly continuous, heats the hot intracluster gas and reduces radiative cooling and subsequent star formation by an order of magnitude. Feedback appears to maintain a long-lived heating/cooling balance. Powerful, jetted radio outbursts may represent a further mode of energy feedback that affects the cores of groups and subclusters. New telescopes and instruments from the radio to X-ray bands will come into operation over the next several years and lead to a rapid expansion in observational data on all modes of AGN feedback.},
	language = {en},
	number = {1},
	urldate = {2019-07-09},
	journal = {Annual Review of Astronomy and Astrophysics},
	author = {Fabian, A.C.},
	month = sep,
	year = {2012},
	pages = {455--489},
	file = {Fabian - 2012 - Observational Evidence of Active Galactic Nuclei F.pdf:/Users/mphf18/Zotero/storage/EA2WFGWE/Fabian - 2012 - Observational Evidence of Active Galactic Nuclei F.pdf:application/pdf}
}


@article{Nelson2015,
	title = {The impact of feedback on cosmological gas accretion},
	volume = {448},
	issn = {1365-2966, 0035-8711},
	url = {http://academic.oup.com/mnras/article/448/1/59/1751796/The-impact-of-feedback-on-cosmological-gas},
	doi = {10.1093/mnras/stv017},
	abstract = {We investigate how the way galaxies acquire their gas across cosmic time in cosmological hydrodynamic simulations is modified by a comprehensive physical model for baryonic feedback processes. To do so, we compare two simulations – with and without feedback – both evolved with the moving mesh code AREPO. The feedback runs implement the full physics model of the Illustris simulation project, including star formation driven galactic winds and energetic feedback from supermassive black holes. We explore: (a) the accretion rate of material contributing to the net growth of galaxies and originating directly from the intergalactic medium, finding that feedback strongly suppresses the raw, as well as the net, inflow of this ‘smooth mode’ gas at all redshifts, regardless of the temperature history of newly acquired gas. (b) At the virial radius the temperature and radial flux of inflowing gas is largely unaffected at z = 2. However, the spherical covering fraction of inflowing gas at 0.25 rvir decreases substantially, from more than 80 per cent to less than 50 per cent, while the rates of both inflow and outflow increase, indicative of recycling across this boundary. (c) The fractional contribution of smooth accretion to the total accretion rate is lower in the simulation with feedback, by roughly a factor of 2 across all redshifts. Moreover, the smooth component of gas with a cold temperature history, is entirely suppressed in the feedback run at z {\textless} 1. (d) The amount of time taken by gas to cross from the virial radius to the galaxy – the ‘halo transit time’ – increases in the presence of feedback by a factor of 2–3, and is notably independent of halo mass. We discuss the possible implications of this invariance for theoretical models of hot halo gas cooling.},
	language = {en},
	number = {1},
	urldate = {2019-07-09},
	journal = {Monthly Notices of the Royal Astronomical Society},
	author = {Nelson, Dylan and Genel, Shy and Vogelsberger, Mark and Springel, Volker and Sijacki, Debora and Torrey, Paul and Hernquist, Lars},
	month = mar,
	year = {2015},
	pages = {59--74},
	file = {Nelson et al. - 2015 - The impact of feedback on cosmological gas accreti.pdf:/Users/mphf18/Zotero/storage/8NPWJZWS/Nelson et al. - 2015 - The impact of feedback on cosmological gas accreti.pdf:application/pdf}
}