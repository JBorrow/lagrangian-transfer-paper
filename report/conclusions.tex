\section{Conclusions}
\label{sec:conclusions}

In this work, a new method of analysis for cosmological simulations has
been presented, along with an open-source analysis framework,
\ltcaesar{}. This method looks primarily at where gas \emph{originated}
in the initial conditions, instead of looking at the final state
of matter for comparisons to observations. This gives us a handle on
key insights into galaxy formation that previously were only speculated
about.

Because of the nature of this work, as a methods paper, there is still
a huge amount of analysis to be performed from a cosmology and galaxy
formation perspective. As this is a report, and not a paper, we refer
the interested reader to \S \ref{sec:futurework} where preliminiary ideas
for this comparison are presented, along with future plans to strengthen
the analysis pipeline.

For now, though, the key results, speculation, and concepts of this paper are
summarised below:

\begin{itemize}
    \item By considering the inter-particle distances of nearest neighbours
    in the initial conditions, it is possible to see that stellar populations
    form out of gas that, generally, is tightly coupled to dark matter. This
    is explained by gas requiring enough time to cool to become dense enough
    to form stars, and this is only possible in the deep potential wells at
    the centre of halos. Gas, on the other hand, can be spread much more,
    relative to the dynamical motions of the background dark matter.
    \item The baryonic matter that resides in any given halo does not necessarily
    follow the same path as the dark matter to get there. It is possible to
    quantify this `inter-lagrangian transfer' by looking at the particles
    that resided in the lagrangian region as defined by the dark matter in the
    initial conditions and comparing this to the final conditions. On average,
    indepdenent of halo mass, only around 60\% of the baryons in a given
    dark matter halo at $z=0$ originated in that lagrangian region.
    Around 10\% of that mass originated in regions defined by the dark matter
    in other halos, with the remaining 30\% originating outside any region
    at all. The former contribution can, possibly, be explained by stellar
    winds and AGN feedback blowing gas out of halos, allowing it to accrete
    onto neighbours. The latter contribution from outside any lagrangian regions
    can be explained by two things: the non-complete filling fraction of the
    lagrangian region definiton chosen in this work, and the strong, $\rho^2$,
    attractor that is provided by cooling. This allows gas to cool and fall
    into the halos quicker than dark matter, which due to a lack of cooling
    cannot lose angular momentum.
    \item Desipte requiring cooling to take place, some small halos can still
    recieve up to a 20\% fraction of their stellar mass from either in-situ
    or ex-situ star formation.
    \item As expected, due to the above observations and explanations, the
    fraciton of baryonic mass from the lagrangian region of a halo decreases
    as a function of radius. As a function of radius, the contribution from
    outside and from other lagrangian regions grows, showing that the
    externally contributed material plays a significant role in the formation
    and dynamics of the CGM.
    \item Several ways to ensure that the analysis is robust, including
    increasing the radius over which material is included in the lagrangian
    region, and smoothing the regions themselves, were presented. These had
    little effects on the overall qualitative results from this work.
\end{itemize}
This area of simulation analysis promises to be a fertile ground for the
future. The authors hope that other simulators interested in such an analysis
will download \ltcaesar{}, and either independently, or with us, run the
same analysis. Feedback models underpin a large amount of the dynamics
involved in this transfer, and hence it is clearly important to compare
various feedback models and modes in future analysis.
