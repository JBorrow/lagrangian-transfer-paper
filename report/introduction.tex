\section{Introduction} \label{sec:introduction} Cosmological simulations have
been used for decades to study the evolution of the universe. Particles or
cells, depending on the choice of numerical method, are tracked over cosmic
time until their final resting place (usually at redshift $z=0$), where the
distribution of matter can be compared to observations. In classic galaxy
formation theory, dark matter collapses early to form virialised halos, into
which gas is also pulled and can cool to form stars at the center
\citep{Mo2010}. Early cosmological simulations were only powerful enough to
include gravitational forces, and the choice was made to only include the
dominant gravitational fluid, dark matter. These dark matter only simulations
\citep[see e.g.][]{Frenk1988, Springel2005a} then had a semi-analytic galaxy
formation model (SAM) applied on top to study the expected properties of bound
objects \citep[see e.g.][for modern examples of SAM frameworks]{Porter2014,
Henriques2015, Lacey2016}. Even these galaxy formation models, to accurately
predict properties of galaxies, require the consideration of baryonic effects.
Feedback from stars and black holes is critical to explain the observed
properties of galaxies. Large-scale winds eject gas from galaxies, which can
re-accrete back, remain in the Circumgalactic Medium (CGM), or reach the
Intergalactic Medium (IGM) outside of halos. This cycling of baryons is an
integral part of modern galaxy formation theory.

It is now possible to run full hydrodynamical models of the universe that
explicitly include the baryonic component. Using codes such as TreeSPH,
RAMSES, and GADGET \citep{Hernquist1989, Teyssier2002, Springel2005b}, along
with full galaxy formation models, such as GEAR, Illustris, and EAGLE
\citep{Revaz2012, Vogelsberger2014, Schaye2015}, it is now
possible to reproduce a large number of observed properties of galaxies.
These codes can include stellar and AGN feedback, star formation, magnetic
fields, and many more physical processes that are believed to be important
for galaxy formation. Recent cosmological `zoom-in' simulations from the FIRE
project \citep{Hopkins2014} have shown that gas ejected in winds from
satellite galaxies can accrete onto the central galaxy, and this
intergalactic transfer of material can be a primary contributor to galaxy
growth. Galaxies providing intergalactic transfer material often end up
merging with the central galaxy, but the extent to which galactic winds can
push gas to larger scales and connect individual central halos at $z=0$
cannot be addressed in `zoom-in' simulations of individual galaxies
\citep{AnglesAlcazar2017}.

In this work, we extend the intergalactic transfer analysis of
\citet{AnglesAlcazar2017} to a large cosmological volume using the \simba{}
simulations \citep{Dave2019}. More generally, we present a framework for
analysing the relative motion of dark matter and baryons on large scales
owing to hydrodynamic and feedback processes. We connect the distribution of
dark matter and baryonic lagrangian resolution elements at $z=0$ with their
original distributions at the initial conditions, identifying the `lagrangian
region' of $z=0$ halos as the region in the initial conditions that will
collapse into each dark matter halo. We quantify for the first time the large
scale gas flows between lagrangian regions and the surrounding IGM and the
importance of `inter-lagrangian transfer' in galaxy evolution.

The remainder of this paper is organised as follows: in \S\ref{sec:simba}, we
discuss the underlying \simba{} simulation suite that is used for analysis;
in \S\ref{sec:feedbackmetrics}, we discuss a distance-based metric for the
investigation of feedback strength; in \S\ref{sec:transfer} we discuss
halo-level metrics based on lagrangian regions to study inter-lagrangian
transfer; in \S\ref{sec:convergence} we discuss the convergence of the method;
and in \S\ref{sec:conclusions} we conclude and summarise the results.