\section{Introduction}
\label{sec:introduction}

Cosmological simulations are an important tool to study the evolution of the
universe. Mass elements of various matter components are tracked over cosmic
time under the influence of gravity and other forces until a desired
redshift, where the distribution of matter can be compared to observations.
The earliest simulations included only dark matter acting under gravity
\citep[see e.g.][]{Frenk1988, Springel2005a}, which remains an important
approach to this day because such simulations are computationally efficient
and can model very large volumes required for, e.g., dark energy studies
\citep{Knabenhans2019}. However, such simulations do not directly model the
observable component. As such, techniques such as semi-analytic models (SAMs)
have been developed \citep{FrenkWhite1990,Kauffmann1996,Somerville1998} to
populate dark matter halos with galaxies \citep[see e.g.][for modern examples
of SAM frameworks]{Porter2014, Henriques2015, Lacey2016}. Crucially, it has
been recognized that feedback processes from the formation of stars and black
holes have an important effect on the resulting observable baryonic
component, though they have a small effect on the collisionless dark matter.
Such feedback often takes the form of large-scale winds that eject
substantial amounts of gas from galaxies owing to energetic input from young
stars, supernovae, and active galactic nuclei (AGN). This gas can then be
deposited far out in the intergalactic medium (IGM), remain as halo gas in
the Circumgalactic Medium (CGM), or be re-accreted as `wind
recycling' \citep{Oppenheimer2010}. This cycling of baryons is an integral
part of modern galaxy formation theory, and is believed to be a key factor in
establishing the observed properties of both galaxies and intergalactic
gas \citep{Somerville2015}.

With advancing computational speed and algorithmic developments, it has
become possible to run full hydrodynamical models of the universe that
explicitly track the baryonic component. %\citep[e.g.][]{Hernquist1989, Teyssier2002, Springel2005b}. 
Beyond modelling hydrodynamical processes,
sub-grid prescriptions have been implemented in order to cool the gas and
produce stars, with increasing levels of refinement and sophistication
\citep[e.g.][]{Revaz2012, Vogelsberger2014, Schaye2015}. Using these models
it is now possible to reproduce many of the key observed properties of
galaxies at a range of cosmic epochs. Modern galaxy formation simulations
typically include radiative cooling, chemical enrichment, star formation,
stellar feedback, and AGN feedback. At times, other (non sub-grid) processes
such as magnetic fields, that are believed to be important for forming
galaxies as observed, are included. Feedback processes are poorly understood,
despite playing a critical role in regulating galaxy growth \citep{Naab2017}.
These models must prevent too much star formation, as well as the `overcooling',
suffered by the earliest hydrodynamical simulations
\citep{Dave2001,Balogh2001}.

Feedback processes also transport baryons far
from their originating dark matter halos. Early observational evidence for
this was that the diffuse intergalactic medium at high redshift is enriched
with metals produced by supernovae, requiring winds with speeds of hundreds
of km/s to be ejected ubiquitously \citep[e.g.][]{Aguirre2001, Springel2003,
 Oppenheimer2006}. More recently, feedback from AGN is seen to eject ionised
and molecular gas outflows with velocities exceeding 1000 km/s
\citep[e.g.][]{Sturm2001, Maiolino2012}. It has long been known that some AGN
also power jets, carrying material out at relativistic velocities. These
processes decouple the baryonic matter from the dark matter on cosmological
scales, which could potentially complicate approaches to populating dark
matter simulations with baryons. Hence it is important to quantify the amount
of baryons that are participating in such large-scale motions, within the
context of modern galaxy formation models that broadly reproduce the observed
galaxy population.

This paper thus examines the large-scale redistribution of baryons relative
to the dark matter, using cosmological simulations that include kinetic
feedback processes which plausibly reproduces the observed galaxy population.
To do this, we pioneer a suite of tools to compare the initial and final
location of baryons relative to their initial `Lagrangian region', defined as
the region in the initial conditions that collapses into a given dark matter
halo. In classical galaxy formation theory, the baryons initially neighbouring
the dark matter would follow the dark matter into the halo, and only then
would significantly decouple owing to radiative processes; this would result
in the baryons lying mostly within its own Lagrangian region. However,
outflows can disrupt this process, and result in transfer outside the
Lagrangian region or even transfer \emph{between} Lagrangian regions. It is
these effects we seek to quantify in this work.

The importance of ejecting baryons and the resulting transfer of material to
other galaxies was highlighted using recent cosmological `zoom-in'
simulations from the FIRE project \citep{Hopkins2014,Hopkins2018}.
Tracking individual gas resolution elements in the simulations, \citet{AnglesAlcazar2017} showed that gas ejected in winds from one galaxy
(often a satellite) can accrete onto another galaxy (often the central) and fuel in-situ star formation. This mechanism, dubbed `intergalactic transfer', was found to be a
significant contributor to galaxy growth. The galaxies
that provided intergalactic transfer material often ended up merging
with the central galaxy by $z=0$, with their mass contribution via winds greatly exceeding that of the merger events.
%and so in some sense the fuelling from satellites only sped up the growth of the central object. In particular,
However, this work did not examine the extent to which galactic winds can push gas to
larger scales and connect individual halos at $z=0$, since it is not feasible to
examine this in zoom-in simulations that by construction focus on modelling a single halo.

In this work, we extend the intergalactic transfer analysis of
\citet{AnglesAlcazar2017} to a large cosmological volume using the \simba{}
simulations \citep{Dave2019}, whose star formation feedback employs scalings
from FIRE, and whose black hole model includes various forms of AGN feedback
including high-velocity jets. More generally, we present a framework for
analysing the relative motion of dark matter and baryons on large scales
owing to hydrodynamic and feedback processes. We quantify for the first time
the large scale gas flows out of Lagrangian regions into the surrounding IGM
and the importance of `inter-Lagrangian transfer' in galaxy evolution.

The remainder of this paper is organised as follows: in \S\ref{sec:simba}, we
discuss the underlying \simba{} simulation suite that is used for analysis;
in \S\ref{sec:feedbackmetrics}, we discuss a distance-based metric for the
investigation of feedback strength; in \S\ref{sec:transfer}, we discuss
halo-level metrics based on lagrangian regions to study inter-lagrangian
transfer; in \S\ref{sec:convergence} we discuss the convergence of the
method; and in \S\ref{sec:conclusions} we conclude and summarise the results.