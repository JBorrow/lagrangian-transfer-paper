\section{Introduction}
\label{sec:introduction}

Cosmological simulations have been used for decades to study the evolution of
the universe. Particles, or cells, are tracked over cosmic time until their
final resting place (usually at redshift $z=0$), where their distributions can
be compared to observations. From the early days where the evolution of a
single dark matter halo could be tracked through to the early 2000s, the focus
was on dark-matter only simulations \citep{frenk1988,
springel_simulations_2005} with semi-analytic models applied on top to study
galaxy formation \citep{porter_2014, henriques_2015, lacey_2016}. Because the
only force that is numerically modeled in these simulations is gravity, it is
possible to define a region in the initial conditions that will collapse to a
given halo in the final state. This is only possible because gravity is a
purely attractive force, allowing energy conservation to be employed, with the
corresponding volume in the initial conditions being referred to as a
`Lagrangian region'.

As the new millenium dawned, cosmologists decided that running a dark-matter
only simulation was no longer enough; computers had become powerful enough to
run full hydrodynamic models using codes such as TreeSPH, RAMSES, and GADGET
\citep{Hernquist1989, teyssier2002, Springel2005}. These codes were adapted
include full galaxy formation models, such as GEAR, Illustris, and EAGLE
\citep{Revaz2011, vogelsberger_properties_2014, Schaye2015}, which can
incorporate stellar and AGN feedback, star formation, magnetic fields, and many
more physical proceses.  These processes break the `only attractive' force
field that was once present in the simulation, making Lagrangian regions less
effective as a diagnostic tool. There is no longer a simple, one-to-one,
relationship between where the matter started, and where it will finish.

Transfer is now possible between lagrangian regions, thanks to these energetic
feedback processes, as well as from outside lagrangian regions, due to the
strong attractor of cooling. The majority of analysis in the past has focused
on where gas ends up, rather than where it originated.
\citet{anglesalcazar2016} looked at individual galaxies in FIRE
\citep{fireproject2014} and how they were able to accrete gas from other nearby
halos, finding that around 50\% of the stellar mass of a Milky-Way mass galaxy
originated from outside. In this work, a similar analysis is extended to an
entire cosmological simulation box, this time only using two snapshots, from
the \simba{} simulation suite \citep{dave2018}. In \S \ref{sec:simba}, the
\simba{} suite is described.  In \S \ref{sec:haloindependent} a simple analysis
based on the distances between particles is considered, with the concept of
lagrangian regions being introduced in \S \ref{sec:lagrangianregions}. In \S
\ref{sec:combining}, these two analysis pathways are combined, and in \S
\ref{sec:conclusions} the conclusions are presented.
