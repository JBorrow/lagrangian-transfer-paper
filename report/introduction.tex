\section{Introduction}

Cosmological simulations have been used for decades to study the evolution of
the universe. Particles, or cells, are tracked over cosmic time until their
final resting place (usually at redshift $z=0$), where their distributions can
be compared to observations. From the early days where single dark matter halos
could be simulated through to the early 2000's the focus was on dark-matter
only simulations with semi-analytic models applied on top to study galaxy
formation. Because the only force that is numerically modeled in these
simulations is gravity, it is possible to define a region in the initial
conditions that will collapse to a given halo in the final state. This is only
possible because gravity is a purely attractive force, allowing energy
conservation to be employed, with the region being referred to as a `Lagrangian
region'.

As the new millenium dawned, cosmologists decided that running a dark-matter
only simulation was no longer enough; computers had become powerful enough to
run full hydrodynamic models using codes such as RAMSES, GADGET, and <>
\citep{}. These codes now also include full galaxy formation models, such as
Illustris, EAGLE, and GEAR \citep{}, which can incorporate stellar and AGN
feedback, star formation, magnetic fields, and many more physical proceses.
These processes break the `only attractive' force field that was once present
in the simulation, making Lagrangian regions less effective as a diagnostic
tool. There is no longer a simple, one-to-one, relationship between where
the matter started, and where it will finish.

Transfer is now possible between lagrangian regions, thanks to energetic 
feedback processes, and from outside lagrangian regions, thanks to the
strong attractor of cooling. The majority of analysis in the past has,
rightly so, focused on where gas ends up, rather than where it originated.
\citet{anglesalcazar2017} looked at individual galaxies in FIRE \citep{fire}
and how they were able to accrete gas from other nearby halos, finding that
around 50\% of the stellar mass of a Milky-Way mass galaxy originated from
outside. In this work, a similar analysis is extended to an entire
cosmological simulation box, this time only using two snapshots, from
the Simba simulation suite \citep{dave2018}.
