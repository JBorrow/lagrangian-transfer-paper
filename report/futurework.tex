\section{Future Work}
\label{sec:futurework}

As noted in \S \ref{sec:conclusion}, this work is preliminary and mainly
focuses on the methods behind this kind of analysis. Below is a list of
possbile opportunities, improvements, and science that we wish to accomplish
using the newly developed \ltcaesar{} code. This code is highly extensible and
can use any halo finder, simulation code (so long as it includes at least
tracer particles), and can run on a $512^3$ particle box in under an hour
(including the production of all of the figures in this report) on a single
supercomputing node. The authors acknowledge that this section, unfortunately,
is under-developed and significantly under-referenced, based on the possible
claims presented here.

\subsection{Improvements to the analysis}

\subsubsection{Better halo-finder support}

In the near future, we hope to add support for completely generic halo-finders.
The majority of halo finders are currently supported through the {\tt caesar}
and {\tt yt} framework, but for simplicity the next addition will be to
include the ability to specify only halo centers and virial radii to perform
the search by hand.

\subsubsection{Virial radius extension work}

The current methodology for increasing the virial radii of halos is not as
clean as it could be. This involves re-finding all particles within $n$ times
$r_{\rm vir}$ and then constructing a new halo catalogue out of this. The
contributions to that halo are then counted within this radius, instead of
to the original virial radius. This means that the extension to the virial
radius is not only filling out lagrangian regions, but increasing their
size as well; this leads to the exact same edge effects that we have seen
previously causing more transfer from outside. In the future, the plan is to
consider particles that lie within $r_{\rm vir}$ at $z=0$ for contribution
to the mass fractions, but include particles within $n \cdot r_{\rm vir}$
in the definition of the lagrangian region itself. This will significantly
smooth the edges of the virial radius.

\subsubsection{Filling-out work}

Currently we fill out the lagrangian regions by using a nearest-neighbour
search in the initial conditions. This works fine, but it would be helpful
to include other definitions, such as the convex hull, for comparison to 
other works and to zoom-in simulations.

\subsubsection{Distance distribution improvements}

The interesting thing about these distance metrics that is still not 
well understood by the authors is the significant distances that dark matter
particles can end up with between them despite starting the simulation
next to each other. These particles may be on either side of a void,
for instance, but it is unclear where the origin of this 7.5 Mpc maximal
distance is (this appears to be independent of resolution and box size
based on preliminary testing).

It should be possible to consider various distance distribution metrics; at the
moment only the nearest neighbour for each dark matter particle is considered.
However, there is much more exploratory work to be done here. Consider, for
instance, the median distance among the nearest $n$ neighbours, the variance in
distance among neighbours as a function of position in the initial conditions.

\subsection{Science work}

\subsubsection{Comparison of models}

In a future paper, once this methods paper is finalised, we hope to re-run
this analysis on \emph{at least} the \mufasa{} and EAGLE simulations. We
also hope to re-run the analysis on the Illustris and Illustris-TNG models,
however this may require a little more code development work due to the
tracer particle model available there for tracking the flows of gas. Ideally,
this should be completely transparent, but this is unclear at the moment.
This work really highlights the inherrent advantages of particle-based
methods, and we hope that it becomes a rataionale behind people using these
(over the often easier to conceptualise eulerian grid methods) in the future.

\subsubsection{The structure of the CGM}

This work shows that the baryonic matter in a given halo, as a function of
radius, has a well-defined origin profile. This should lead to a well-defined
metallicity profile. These can be investigated using the \simba{} simulations
and matched to observations.

This external contribution may also lead to a different density profile than
that is expected from analytical and semi-analytical work. This differing
density profile can then be broken into components based on their origin,
thanks to this work. This could affect many cosmological probes, including
weak-lensing measurements that are used to constrain modified gravity models.

\subsubsection{Zoom-in simulations}

This work should have some implications for zoom-in simulations. Current zoom-ins
neglect sub-grid physics for particles outside of the high-resolution zone,
but based on this analysis it appears that this may no longer be sufficient
to capture the full physics; particles can travel up to 15 Mpc from their
initial neighbour, and significant transfer is seen between galaxies. This is
especially important in work that aims to capture the physics of the CGM in
those galaxies.
