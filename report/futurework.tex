\section{Future Work}
\label{sec:futurework}

As noted in \S \ref{sec:conclusions}, this work is preliminary and mainly
focuses on the methods behind this new approach. Below is a list of possible
opportunities, improvements, and science that we wish to accomplish using the
newly developed \ltcaesar{} code.

\subsection{Improvements to the analysis}

\subsubsection{Improvements to particle ID handling}

Due to a bug in the simulation code when this particular box was simulated,
gas particles which have formed a star, and star particles that formed out of
a gas particle which had already created a star particle, must be excluded.
This is unfortunate, as these particles are the ones that would probably have
extended out to the largest distances due to their interactions with stellar
and AGN feedback. In future analysis this problem will be fixed; however at
this time we do not expect it to have a significant effect as this is a very
small contribution (less than 0.1\% of gas particles).

\subsubsection{Filling-out work}

Currently we fill out the lagrangian regions by using a nearest-neighbour
search in the initial conditions. This works fine, but it would be helpful
to include other definitions, such as the convex hull, for comparison to 
other works and to zoom-in simulations.

\subsubsection{Distance distribution improvements}

The interesting thing about these distance metrics that is still not 
well understood by the authors is the significant distances that dark matter
particles can end up with between them despite starting the simulation
next to each other. These particles may be on either side of a void,
for instance, but it is unclear where the origin of this 7.5 Mpc maximal
distance is (this appears to be independent of resolution and box size
based on preliminary testing).

It should be possible to consider various distance distribution metrics; at the
moment only the nearest neighbour for each dark matter particle is considered.
However, there is much more exploratory work to be done here. Consider, for
instance, the median distance among the nearest $n$ neighbours, the variance in
distance among neighbours, and this metric as a function of position in the
initial conditions.

\subsection{Science work}

\subsubsection{Comparison of models}

In a future paper, once this methods paper is finalised, we hope to re-run
this analysis on \emph{at least} the \mufasa{} and EAGLE simulations. We
also hope to re-run the analysis on the Illustris and Illustris-TNG models,
however this may require a little more code development work due to the
tracer particle model available there for tracking the flows of gas. Ideally,
this should be completely transparent, but this is unclear at the moment.
This work really highlights the inherrent advantages of particle-based
methods, and we hope that it becomes a rationale behind people using these
(over the often easier to conceptualise eulerian grid methods) in the future.

\subsubsection{The structure of the CGM}

This work shows that the baryonic matter in a given halo, as a function of
radius, has a well-defined origin profile. This should lead to a well-defined
metallicity profile. These can be investigated using the \simba{} simulations
and matched to observations.

This external contribution may also lead to a different density profile than
that is expected from analytical and semi-analytical work. This differing
density profile can then be broken into components based on their origin,
thanks to this work. This could affect many cosmological probes, including
weak-lensing measurements that are used to constrain modified gravity models.

\subsubsection{Zoom-in simulations}

This work should have some implications for zoom-in simulations. Current zoom-ins
neglect sub-grid physics for particles outside of the high-resolution zone,
but based on this analysis it appears that this may no longer be sufficient
to capture the full physics; particles can travel up to 15 Mpc from their
initial neighbour, and significant transfer is seen between galaxies. This is
especially important in work that aims to capture the physics of the CGM in
those galaxies.
