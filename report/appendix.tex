\section{Appendix}
\label{sec:appendix}

\subsection{Code}
\label{app:code}

The code used in this analysis, \ltcaesar{} is made available to the
community for use with their own
simulations\footnote{https://www.github.com/jborrow/lagrangian-transfer}.
This code can also be installed with the `{\tt pip}' python packaging
manager. In the repository, there is also a series of scripts that can be
used to convert between halo catalogues, and a robust suite of unit tests
that ensure that \ltcaesar{} performs the analysis that we claim it does.
This code is highly extensible and can use any halo finder, simulation code
(so long as it includes at least tracer particles), and can run on a $512^3$
particle box in under an hour (including the production of all of the figures
in this report) on a single supercomputing node. The authors acknowledge that
this section, unfortunately, is under-developed and significantly
under-referenced, based on the possible claims presented here.

\ltcaesar{} uses the numerical routines from {\tt numpy}, the KDTree from
{\tt scipy} for nearest neighbour searching, and the halo finder wrapper from
{\tt caesar} \citep{NumPy2018, Jones2001, Thompson2018}. Full documentation,
and more information, can be found on the code webpages. The work in this paper
in particular made use of the Intel Distribution for Python that provides
optimized routines for {\tt numpy} and {\tt scipy} \citep{Pavlyk2017}.

The visualisations in this work made use of {\tt py-sphviewer}
\citep{Benitez-Llambay2015}.

