\section{Lagrangian Regions}
\label{sec:lagrangianregions}

\subsection{Definition of a lagrangian region}

\begin{figure}
    \centering
    \includegraphics[width=\columnwidth]{figures/lagrangian_region_high_mass_mnras.pdf}
    \caption{The lagrangian region associated with a $7\times10^{13}$
	$\mathrm{M}_\odot$ halo at $z=0$, shown in the inset plot. The local
	particle density in this region is approximately constant; the
	structure shown here is due to the particle selection, rather than any
	real structure in the overall distribution. Colour encodes projected
	density.}
    \label{fig:lrpic}
\end{figure}

A lagrangian region is defined as the region in the initial conditions where
the dark matter from a given collapsed object at lower redshift resides. The
following discussion describes comparison with redshift $z=0$ compact objects
but this definition is easily extensible to higher redshift.

To extract the collapsed objects at $z=0$ the AMIGA halo finder
\citep[AHF,][]{ahfi, ahfii} is used. This spherical overdensity finder
determines the halo centers by using a nested grid, and then fits parameters
based on the Navarro-Frenk-White \citep[NFW, ][]{nfw} profile.

\subsection{Matching lagrangian IDs}

Each dark matter particle in the simulation is assigned, based on the known
unique particle ID, a lagrangian region identifier that corresponds to the halo
ID of the associated $z=0$ collapsed object. Particles which end up outside of
any halo at $z=0$ are assigned a lagrangian region identifier of -1. 
To extend the definition of the lagrangian region from dark matter to the
baryonic particles, a nearest neighbour search for each gas
particle at $z=z_{\rm ini}$ is performed and each particle assigned a
lagrangian region identifier that corresponds to that of the nearest dark
matter particle. This ensures that the very fine-grained detail present in the
lagrangian region is preserved (see Figure \ref{fig:lrpic}).

Once all particles in the initial conditions have been assigned a lagrangian
region, they must be ID matched with particles in the final, $z=0$, snapshot of
the simulation. This is performed by looping through all of the (sorted)
particles and assigning a lagrangian region ID for the final-state particles
that is equal to that of their initial state progenitor. To assign a
lagrangian region to the star particles at $z=0$, their gas progenitor
(which can be tracked using the unique particle ID) is used. Particles which
either become black holes, or are consumed by a black hole, are ignored for
this analysis. This re-matching also takes place for the dark matter
particles, to ensure that the ID matching is working correctly.

\subsection{Quantifying inter-lagrangian transfer}

\begin{figure} \centering
\includegraphics[width=\columnwidth]{generated_figures/basic_halos/component_fraction_vs_halo_mass_both.pdf}
 \caption{The fraction of total baryonic mass in a given halo at $z=0$ as a
 function of halo mass, originating from the lagrangian region defined by the
 halo (blue), from outside any lagrangian region (purple), and from the
 lagrangian region defined by another halo (red). These values are computed
 using the lagrangian region IDs that were assigned to the gas and star
 particles in the initial conditions. See Figure \ref{fig:splitmassfrac} for
 a breakdown into stellar and gaseous components. The shaded regions show a
 single standard deviation of variance in the mass fraction for that bin, and
 do not include errors from halo sampling bias or cosmic variance.}
 \label{fig:massfrac}
 \end{figure}

Once this analysis has been performed it is possible to calculate the fraction of
baryonic mass at $z=0$ that originates from the lagrangian region of a given
halo (see Figure \ref{fig:massfrac}). There is a significant difference in the
contributions from the gaseous and stellar components to this mass fraction;
see Figure \ref{fig:splitmassfrac}. This data is for every halo in the box, and
hence does not include any cuts based on the particular neighbourhood of these
halos; a significant fraction of the scatter here is likely to come from
isolated halos, versus those in clusters and other noisier environments. This
analysis shows that a significant portion (up to 20\%) of the stellar mass of a
Milky-Way mass halo may come from the lagrangian region as defined by a
\emph{different halo}.

\begin{figure*} \centering
	\includegraphics[width=0.495\textwidth]{generated_figures/basic_halos/component_fraction_vs_halo_mass_gas.pdf}
	\includegraphics[width=0.495\textwidth]{generated_figures/basic_halos/component_fraction_vs_halo_mass_stellar.pdf}
	\caption{Left: fraction of gaseous mass at $z=0$ in each halo from each
	component; right: fraction of stellar mass at $z=0$ from each
	component. Note that there is significantly more transfer shown in the
	gaseous component. Gas that is transferred between lagrangian regions
	must be given time to cool before being able to form stars. As the
	events that enable transfer are typically very energetic (AGN, stellar
	feedback, accretion), it is unlikely that the cooling time will be
	short enough to form stars by the end of the simulation for most
	transfer.} \label{fig:splitmassfrac} \end{figure*}

It is important to note that this spread in mass fractions as a function of
halo mass is still to be quantified. It could be that those halos which
end up having less mass transfer are those which are more isolated; in future
analysis we hope to include the isolation criteria used in, for example, the
APOSTLE project \citep{fattahi2016} to select halos and compare their mass
fractions. This inter-lagrangian transfer could have a significant effect
on these high-resolution zoom-in galaxies that is not correctly captured
using the current isolation criteria.

\subsection{Radial trends}

\begin{figure} \centering
	\includegraphics[width=\columnwidth]{generated_figures/basic_halos/radial_distance_plot.pdf}
	\caption{The dependence of mass fraction split by component as a
	function of radius, normalised by the virial radius for each halo.
	Every halo with $10^{12} \leq $ M$_{\rm halo}$ $ < 10^{13}$ M$_\odot$
	is stacked in this plot, with the shaded regions giving standard errors
	around the mean. Solid lines show the trends for gas, with the dashed
	lines showing the same but for the stellar component of the halo. 50
	bins were spaced linearly in radius.} \label{fig:radialmassfrac}
\end{figure}

The mass fractions contributed from various components to each halo appear to
be relatively independent of halo mass (see Figure \ref{fig:massfrac}). In
Figure \ref{fig:radialmassfrac} the mass fraction contributed to the halo by
each component is shown as a function of radius. As expected, more gas in the
center of the halo comes from the corresponding lagrangian region, but
interestingly this only approaches 70\%, suggesting significant transfer
\emph{into halos} still takes place for gas that ends up at the bottom of the
potential well at $z=0$.

Also note how the mass fraction of stars originating from the lagrangian
region defined by each halo drops as a function of radius. Note that around
$10$\% of the stars at the center of these halos (within 0.1\rvir{})
are formed from gas that was not present in the initial lagrangian region of
this halo. This is surprising, as this transfer must have taken place
relatively early to ensure that the gas was able to cool and form stars.

These radial trends tell us something about the assembly of the
Circumgalactic Medium (CGM) \citep{tumlinson2017}. These trends are very
tight and show strong convergence when multiple halos are stacked.
\citet{Crain2013} found that the CGM assembles from the `inside-out', with
feedback within the galaxy establishing a strong negative metallicity
gradient. However, these results seem to suggest there is a significant
effect from inter-lagrangian (and hence inter-galactic) transfer from gas
that has been blown out of \emph{other} galaxies falling in to add to this
metallicity gradient. Future work will focus on separating these two
variables using this analysis.
