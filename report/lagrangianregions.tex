\section{Lagrangian Regions}

\subsection{Definition of a Lagrangian Region}

\begin{figure}
    \centering
    \includegraphics[width=\columnwidth]{figures/lagrangian_region_visualisation.png}
    \caption{The lagrangian region associated with a $7\times10^{13}$ $\mathrm{M}_\odot$ halo at $z=0$, shown in the inset plot. The local particle density in this region is approximately constant; the structure shown here is due to the particle selection, rather than any real structure in the overall distribution. Colour encodes projected denisity.}
    \label{fig:lrpic}
\end{figure}

A lagrangian region is defined as the region in the initial conditions where the dark matter from a given collapsed object at lower redshift resides. This definition is somewhat open to interpretation; there may be particles enclosed in the convex hull of such a region that do not end up in the collapsed object, especially in an incorrect choice is made for the gravitational softening. The following discussion describes comparison with redshift $z=0$ compact objects but this definition is easily extensible to higher redshift.

To determine the collapsed objects at $z=0$ the AMIGA halo finder \citep[AHF][]{ahf_refernces} is used. This spherical overdensity finder determines the halo centers by using a nested grid, and then fits parameters based on the Nevarro-Frenk-White \citep[NFW, ][]{nfw_ref} profile. The results presented in the remainder of the work use this halo finder unless specified, and a comparison with a more traditional three-dimensional friends-of-friends algorithm is given in Appendix \ref{app:caesar}.

It is impractical, especially considering the nature of this paper, to use the baryonic matter resident in a collapsed object to define the appropriate `baryonic' lagrangian region. Instead, a nearest neighbour search for each gas particle at $z=z_{\rm ini}$ is performed and each particle assigned a lagrangian region identifier that corresponds to that of the nearest dark matter particle. This ensures that the two regions overlap very tightly spatially, which is important when considering the very fine-grained detail present in these regions (see Figure \ref{fig:lrpic}).

\blue{If we don't fix the ID issue we should definitely mention that here.}

\subsection{Matching Lagrangian IDs}

Each dark matter particle in the simulation is assigned, based on the known unique particle ID, a largangian region identifier that cooresponds to the halo ID of the associated $z=0$ collapsed object. Particles which end up outside of any halo at $z=0$ are assigned a lagrangian region identifier of -1. Once this has been performed, the gas particles may be matched as described above, with the nearest neighbour search. Once all particles in the initial conditions have been assigned a lagrangian region, they must be ID matched with particles in the final $z=0$ snapshot of the simulation. This is performed by looping through all of the (sorted) particles and assigning a lagrangian region ID for the final-state particles that is equal to that of their initial state progenitor. To assign a largrangian region to the star particles at $z=0$, their gas progenitor (which can be tracked using the unique particle ID) is used. Particles which either become black holes, or are consumed by a black hole, are ignored for this analysis. This re-matching also takes place for the dark matter particles, to ensure that the ID matching is working correctly.

\blue{Perhaps a flow-chart might be nice here to make this a little clearer.}

\subsection{Quantifying Inter-Lagrangian Transfer}

\begin{figure}
    \centering
    \includegraphics[width=\columnwidth]{figures/overall_mass_fraction.png}
    \caption{The fraction of total baryonic mass in a given halo at $z=0$ as a function of halo mass. These values are computed using the lagrangian region IDs that were assigned to the gas and star particles in the initial conditions. See Figure \ref{fig:splitmassfrac} for a breakdown into stellar and gaseous components. The shaded regions show a single standard deviation of variance in the mass fraction for that bin, and do not include errors from halo sampling bias or cosmic variabnce. \blue{Might be worth including those errors in our analysis? I presume this is why our errors drop off as a function of halo mass, rather than this actually being an effect of the scatter getting lower.}}
    \label{fig:massfrac}
\end{figure}

Once this analysis has been performed it is possible to see the fraction of baryonic mass at $z=0$ that originates from the lagrangian region of a given halo (see Figure \ref{fig:massfrac}). There is a significant difference in the contributions from the gaseous and stellar components to this mass fraction; see Figure \ref{fig:splitmassfrac}. This data is for every halo in the box, and hence does not include any cuts based on the particlar neighbourhood of these halos; a significant fraction of the scatter here is likely to come from isolated halos, versus those in clusters and other noisier environments. This analyis shows that a significant portion (up to 20\%) of the stellar mass of a Milky-Way mass halo may come from the lagrangian region as defined by a \emph{different halo}.

\blue{It would be nice at some point to include isolation criteria here. After speaking to some people who do zoom-ins of clusters, it seems that they actually only generate hydrodynamics particles that are within the LR (as defined by the dark matter out to ~3 Mpc of the halo of interest). If we see significant transfer even for isolated, milky way mass, galaxies, that would really break things like APOSTLE. Thankfully at the moment they have chosen very isolated halos, though.}

\begin{figure*}
    \centering
    \includegraphics[width=0.495\textwidth]{figures/gas_mass_fraction.png}
    \includegraphics[width=0.495\textwidth]{figures/stellar_mass_fraction.png}
    \caption{Left: fraction of gaseous mass at $z=0$ in each halo from each component; right: fraction of stellar mass at $z=0$ from each component. Note that there is significantly more transfer shown in the gaseous component. Gas that is transferred between lagrangian regions must be given time to cool before being able to form stars. As the events that enable transfer are typically very energetic (AGN, stellar feedback, accretion), it is unlikely that the cooling time will be short enough to form stars by the end of the simulation for most transfer.}
    \label{fig:splitmassfrac}
\end{figure*}

\subsection{Redshift Evolution}

\blue{We should re-run the above analysis at a lower redshift, just to check that the transfer is actually lower, like we already had before the disaster.}

\subsection{Radial Trends}

\begin{figure}
    \centering
    \includegraphics[width=\columnwidth]{figures/gas_mass_fraction_radius.png}
    \includegraphics[width=\columnwidth]{figures/star_mass_fraction_radius.png}
    \caption{Top: Gas mass fraction from each component as a function of radius, scaled by virial radius. Bottom: the same plot but for the stellar component. This is shown for all halos with $10^{13} > M_{\rm halo} > 10^{12}$ $M_\odot$. Bins are spaced linearly in radius and stacked for all halos, with bins in individual halos containing no gas or stars being ignored during the normalisation process. The trends steepen with decreasing halo mass, and flatten with increasing mass.}
    \label{fig:radialmassfrac}
\end{figure}

The mass fractions contributed from various components to each halo appear to be relatively independent of halo mass (see Figure \ref{fig:massfrac}), however this may not be true within each individual halo. In Figure \ref{fig:radialmassfrac} the mass fraction contributed to the halo is shown as a function of radius. As expected, more gas in the center of the halo comes from the corresponding lagrangian region, but interestingly this only approaches ~60\%, suggesting significant transfer \emph{into halos} still takes place for gas that ends up at the bottom of the potential well at $z=0$.

Also note how the mass fraction of stars from the own lagrangian region of the halo drops as a function of radius. This suggests that a large number (~30\%) of these stars were formed ex-situ. \blue{Perhaps this is a good place to reference other work... This is a surprisingly high number}. Note, again, that around ~15\% of the stars at the center of these halos are formed from gas that was not present in the initial lagrangian region of this halo. This is surprising, as this transfer must have taken place relatively early to ensure that the gas was able to cool and form stars.
