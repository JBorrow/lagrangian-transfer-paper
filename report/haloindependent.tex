\section{Halo-indepdent Measures of Transfer}
\label{sec:haloindependent}

\subsection{Using Inter-Particle Distances to Describe Transfer}

\begin{figure*} \centering
	\includegraphics[width=\textwidth]{figures/kspafig.pdf} \caption{A
	diagramatic representation of the distance measure. On the left, the
	initial conditions are shown. The blue dark matter particles each find
	their closest dark matter and gas (red) neighbour. These particles are
	then tracked to the final state of the simulation (right) and the
	distances between them calculated again. \blue{In the final version of
	this plot, I would like to re-make it such that this is completely
	vertical and the two plots are stacked on top of each other. Then this
	would just take up a single column. Also $r_{gas}$ and $r_{DM}$ need to
	be shown in the final conditions.}} \label{fig:distancemeasure}
\end{figure*}

As mentioned above, Lagrangian regions are one way to connect the final and
initial state of matter in a simulation. Such an analysis, however, depends on
the definition of a halo, which is unecessary. If the gas and dark matter
become decoupled, it should be possible to see this effect without having to
consider bound structues at all.

To find how separated the dark matter becomes from the gas, two distances are
considered. First, the distance from each dark matter particle to the
corresponding closest dark matter neighbour in the initial conditions, at
$z=0$, i.e. the distance \begin{equation} r_{ij, ~z=0} = \sqrt{ \left|
\mathbf{x}_{i, ~z=0} - \mathbf{x}_{j \ni \min(r_{ij, ~z=z_{ini}}), ~z=0}
\right|^2 } \label{eqn:minimal} \end{equation} where $\mathbf{x}_i$ is the
position of particle $i$, and $\mathbf{x}_i - \mathbf{x}_j$ is wrapped within
the periodic box. The corresponding distance between the dark matter particle
and the cloest gas neighbour in the initial conditions is also considered,
$r_{\rm gas}$. A diagramatic representation of this is given in Figure
\ref{fig:distancemeasure}. As in the halo-based analysis, star particles at
$z=0$ are ID matched with their gas progenetors.

\subsection{Particle-by-Particle Comparison to Dark Matter}

Now that the distances at $z=0$ have been identified, it is possible to compare
how much the dark matter has been able to decouple from itself, compared to the
gas. For each particle, the distance that the nearest dark matter neighbour has
travelled, compared to the nearest gas neighbour, is plotted in Figure
\ref{fig:dmvsstarvsgas}. The stellar distribution is highly symmetric and peaks
around $r_{\rm star}/r_{\rm DM} = 1$, implying that the stellar and dark matter
components have a very similar dynamical distribution (see also Figure
\ref{fig:alldistances}) but that this is not a \emph{local} effect. The gas
that forms stars and dark matter become decoupled differently from their
nearest neigbour, meaning that this is unlikely to be due to, for example, two
particles ending up on different sides of a void.

\begin{figure} \centering
	\includegraphics[width=\columnwidth]{figures/dm_vs_stars_vs_gas.png}
	\caption{The distribution of relative distances for each dark matter
	particle to the appropriate gas and stellar particles at $z=0$
	normalised by the distance to the closest dark matter particle in the
	initial conditions. Note the significantly diffrent distributions for
	gas and stars.} \label{fig:dmvsstarvsgas} \end{figure}

The same distribution for the gas, however, has a different signature. $r_{\rm
gas}/r_{\rm DM}$ peaks around 10, not 1, showing that gas ends up
preferrentially further away than the neighbouring dark matter particle. This
is unsurprising; gas particles feel repulsive forces from hydrodynamics, can be
heated, and even get blown out of galaxies. All of this would work to prevent
gas from falling into the center of a potential well as far as the given dark
matter neighbour. The further-right peak in this plot also substantiates the
claim made in Figure \ref{fig:splitmassfrac} that there is significantly more
transfer of gas than stellar material. The extreme width of this and the
distribution (around 8-9 orders of magnitude) shows that the gravitational
dynamics still plays a significant role.

\subsection{Distribution of Distances}

Instead of comparing on a particle-by-particle basis, let us now consider the
full distribtuion of distances. In Figure \ref{fig:alldistance} the similar
dynamical distributions of the stellar and dark matter components are shown,
with the gaseous component having a significantly longer tail. By the end of
the simulation, gas particles can end up around 15 Mpc away from their original
nearest neighbour; the only way this can occur is due to the strong wind
velocities that are powered by the AGN in the simulation.

\begin{figure} \centering
	\includegraphics[width=\columnwidth]{figures/all_distances.png}
	\caption{The distribution of all distances to particles at $z=0$. Note
	the similarity between the dark matter distribution and the stellar
	distribution, and how different they are to the gas with the associated
	long tail. The gas that ends up in this long tail will have been blown
	out by AGN and ends up outside halos in large superbubbles; see Figure
	\ref{fig:}.} \label{fig:alldistances} \end{figure}

