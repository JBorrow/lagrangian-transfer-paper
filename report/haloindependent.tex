\section{Halo-indepdent Measures of Transfer}
\label{sec:haloindependent}

\begin{figure*} \centering
	\includegraphics[width=\textwidth]{figures/kspafig.pdf} \caption{A
	diagramatic representation of the distance measure. On the left, the
	initial conditions are shown. The blue dark matter particles each find
	their closest dark matter and gas (red) neighbour. These particles are
	then tracked to the final state of the simulation (right) and the
	distances between them calculated again.} \label{fig:distancemeasure}
\end{figure*}

\subsection{Using inter-particle distances to describe transfer}

Usually, analysing how baryonic and dark matter move differently requires the
use of a halo finder, to identify structures between which gas can flow.
However, if the gas and dark matter become decoupled, it should be possible to
see this effect without having to consider bound structues at all.

To find how separated the dark matter becomes from the gas, two distances
\emph{in the final snapshot at $z=0$}\emph{in the final snapshot at $z=0$}  are
considered. First, the distance from each dark matter particle to the
corresponding closest dark matter neighbour in the \emph{initial conditions},
i.e. the distance
\begin{equation}
    r_{ij, ~z=0} = \sqrt{ \left| \mathbf{x}_{i,
    ~z=0} - \mathbf{x}_{j \ni \min(r_{ij, ~z=z_{ini}}), ~z=0} \right|^2 }
    \label{eqn:minimal}
\end{equation}
where $\mathbf{x}_i$ is the position of particle $i$, and $\mathbf{x}_i -
\mathbf{x}_j$ is wrapped within the periodic box. The corresponding distance
between the dark matter particle and the cloest gas neighbour in the initial
conditions is also considered, $r_{\rm gas}$. A diagramatic representation of
this is given in Figure \ref{fig:distancemeasure}. Star particles at $z=0$ are
ID matched with their gas progenetors, where at all possible. Due to a bug in
the simulation code when this particular box was simulated, gas particles which
have formed a star, and star particles that formed out of a gas particle which
had already created a star particle, must be excluded. This is unfortunate, as
these particles are the ones that would probably have extended out to the
largest distances due to their interactions with stellar and AGN feedback. In
future analysis this problem will be fixed; however at this time we do not
expect it to have a significant effect as this is a very small contribution
(less than 0.1\% of gas particles).

\subsection{Particle-by-particle comparison to dark matter}

\begin{figure} \centering
	\includegraphics[width=\columnwidth]{figures/dm_vs_stars_vs_gas.png}
	\caption{The distribution of relative final state distances for each
	dark matter particle to the appropriate gas and stellar (these
	correspond to $i$ in the bottomlabel) particles at $z=0$ normalised by
	the distance to the dark matter particle that was closest in the
	initial conditions. Note the significantly diffrent distributions for
	gas and stars.}
\label{fig:dmvsstarvsgas} \end{figure}

Now that the distances at $z=0$ have been identified, it is possible to compare
how much the dark matter has been able to decouple from itself, compared to the
gas. For each particle, the distance that the nearest dark matter neighbour has
travelled, compared to the nearest gas neighbour, is plotted in Figure
\ref{fig:dmvsstarvsgas}. The stellar distribution is highly symmetric and peaks
around $r_{\rm star}/r_{\rm DM} = 1$, implying that the stellar and dark matter
components have a very similar dynamical distribution (see also Figure
\ref{fig:alldistances}) but that this is not a \emph{local} effect. The gas
that forms stars and dark matter become decoupled differently from their
nearest neigbour, meaning that this is unlikely to be due to, for example, two
particles ending up on different sides of a void.

The same distribution for the gas, however, has a different signature. $r_{\rm
gas}/r_{\rm DM}$ peaks around 10, not 1, showing that gas ends up
preferrentially further away than the neighbouring dark matter particle. This
is unsurprising; gas particles feel repulsive forces from hydrodynamics, can be
heated, and even get blown out of galaxies. All of this would work to prevent
gas from falling into the center of a potential well as far as the given dark
matter neighbour. The further-right peak in this plot also substantiates the
claim made in Figure \ref{fig:splitmassfrac} that there is significantly more
transfer of gas than stellar material. The extreme width of this and the
distribution (around 8-9 orders of magnitude) shows that the gravitational
dynamics still plays a significant role.

The specific details of the dynamics that causes this spread is still not
understood, and must be investigated in further work. This very well-defined
power-law distribution seen in the dark matter must have some signature within
the simulation itself.

\subsection{Distribution of Distances}

\begin{figure} \centering
	\includegraphics[width=\columnwidth]{figures/all_distances.png}
	\caption{The distribution of all distances to particles at $z=0$. Note
	the similarity between the dark matter distribution and the stellar
	distribution, and how different they are to the gas with the associated
	long tail.} \label{fig:alldistances}
\end{figure}

Instead of comparing on a particle-by-particle basis, let us now consider the
full distribtuion of distances. In Figure \ref{fig:alldistances} the similar
dynamical distributions of the stellar and dark matter components are shown,
with the gaseous component having a significantly longer tail. By the end of
the simulation, gas particles can end up around 15 Mpc away from their original
nearest neighbour; the only way this can occur is due to the strong wind
velocities that are powered by the AGN in the simulation.

Such a large separation is certainly possible over the course of the simulation
in the \simba{} model. AGN winds are powered at around $10^4 \kms{}$, meaning
that over the whole run-time of the simulation the maximal distance that
a wind can travel is nearly 150 Mpc; enough to wrap the whole box three times.

The similarity between the dark matter and stellar distribution is also clear
here. Both follow extremely similar power-laws. Again, this is something that
must be investigated further in the future. This could possibly be a signature
of the gravitational softening (hence leading to the similar distributions as
stellar and dark matter particles are treated the same by the simulation code).
It could also be a signature of tidal stripping of satellite halos, now thought
to be a significant effect, as was shown by \citet{vandenbosch2018}.

